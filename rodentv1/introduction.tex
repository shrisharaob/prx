\section{\label{sec:level1}Introduction}
Orientation selectivty has been experimentally observed in the neurons of the primary visual cortex (V1) of various species. The cortical organization of neurons in some of these species like the primates exhibit orientation maps. In the theories prevalent of OS, such an arrangement of neurons is often associated with functional connectivity, i.e. neurons with similar preferred orientation have a higher probability of connections. The necessity of such a connectivity scheme has seldom been questioned\@. 
Carl van Vreeswijk and Haim Somplolinsky \cite{} showed that  under very general conditions, a random network of binary neurons with excitatory and inhibitory synapses automatically converges to a fixed point where the total excitatory and inhibitory currents cancel each other on average. This fixed point is the balanced state. In this state, both the mean and fluctuation of the total input is of the order of the threshold\@. \\
In section \ref{sec:level2}, we analytically demonstrate that OS can emerge in a random network of binary neurons without any functional connectivity, provided the network is operating in the balanced regime. Experimental data from the rodents provides further support to our theory. Alghough most members of the rodent family(order) lack orientation maps, strong OS is observed in the V1 of these animals. A previous study demonstrated this mechanism through numerical simulations of a network of conductance based neurons. \cite{}\@. 
Experimental results indicate significant changes occuring in the selectivity of rodent V1 neurons during the critical period(P23 - 30). After maturation, cortical synaptic weights exhibit long-tailed distributions with only a small number of strong synapses. In section \ref{sec:specific_con}, we will study the effects of introducing a very small number of funcionally specific connections in a ring model operating in the balanced regime. Under certain conditions as shall be described, this leads to spontaneous symmetry breaking. Finally, in section \ref{sec:simulations}, we will present the results of finite N simulations and compare it to the theory. 
 
%%% Local Variables:
%%% mode: latex
%%% TeX-master: "main"
%%% End:
