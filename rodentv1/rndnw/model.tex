We model a network of neurons that are selective for a 1-dimensional stimulus 
feature $\theta$\@. To avoid the difficulty of having to deal with what happens
at the boundary we assume that $\theta$ is a periodic  variable with period 
$2\pi$\@.. 
The network consists of $N_E$ excitatory and $N_I$ inhibitory 
neurons organized on a $2\pi$ periodic ring. We denote neuron $i$ of 
population $a$ (with $a=E,I$ and $i=1,2,\ldots,N_a$) as neuron $(i,a)$. 
The position on the ring of neuron $(i,a)$ is $\phi_i^a=2\pi i/N_a$.

The neurons are modeled as sequentially updated
binary units. The state, $\sigma_i^a$, of neuron $(i,a)$ is updated, 
on average, once per time constant $\tau_a$\@. It is set to zero if, at the 
time of the update, its input, $u_i^a$, is below the threshold, $T_a$, 
and set to one otherwise.

The input, $u_i^a$, has three components: the feedforward input, $u_i^{a0}$, 
from excitatory units in the input layer, the recurrent excitatory input, 
$u_i^{aE}$, and the recurrent inhibitory feedback, $u_i^{aI}$\@. 

The recurrent excitation and inhibition are given by
\begin{equation}
u_i^{ab}(t)=\frac{J_{ab}}{\sqrt{K}}\sum_{j=1}^{N_b}C_{ij}^{ab}\sigma_j^b(t)
\qquad \qquad (b=E,I),
\end{equation}
where $J_{ab}/\sqrt{K}$, with $J_{aE}>0$ and $J_{aI}<0$, is the contribution 
of an active presynaptic cell to the recurrent input.
The connection matrix $C_{ij}^{ab}$ is randomly chosen: $C_{ij}^{ab}=1$ 
with probability $P_{ij}^{ab}=\frac{K}{N_b}[1+2p_{ab}\cos (\phi_i^a-\phi_j^b)]$
and $C_{ij}^{ab}=0$ otherwise. Thus, on average a neuron receives recurrent 
input from $K$ excitatory and $K$ inhibitory units. The modulation, $p_{ab}$, 
$0\leq p_{ab} \leq 1/2$, determines how much the connection probability from 
population $b$ to population $a$ depends on the distance between neurons on 
the ring. Note that, for convenience in the analysis we have assumed that the 
distance dependence of the connection probability is given by 
$1+p\cos(\Delta \phi)$. However, other unimodal probability profiles that are 
symmetric and have their peak at $\Delta\phi=0$ show qualitatively similar 
behavior.

The neurons in the network receive feedforward input, $u_i^{a0}$, from 
excitatory neurons in the input layer.
We assume that there is a pool of $N_0$ of such excitatory neurons 
whose average activity, $m_i^0$, is given by
\begin{equation}
m_i^0=m_0[1+\mu\cos (\theta-\phi_i^0)]\equiv m_0(\theta-\phi_i^0).
\end{equation}.
Here $\theta$ is the feature value of the stimulus and $\phi_i^0=2\pi i/N_0$ 
is the preferred feature value of neuron $(i,0)$\@. The average activity, 
$m_0$, is an increasing function of the stimulus intensity and $\mu$ determines 
how much the response of the input cells is modulated with the stimulus feature.

We consider two models for the pool of input units. In the first the activity,
$\sigma_u^0$ is a continous variable that does not change over time, 
$\sigma_i^a(t)=m_i^a$\@. In the second model the input units are binary 
variable that are sequentially updated, with each cell updated on average once 
per time constant $\tau_0$, at the up[date $\sigma_i^0$ is set to 1 with 
probability $m_i^0$ and set to zero otherwise.

The feedforward input, $u_i^{a0}$, is given by
\begin{equation}
u_i^{a0}(t)=\frac{J_{a0}}{c\sqrt{K}}\sum_{j=1}^{N_0}C_{ij}^{a0}\sigma_j^0(t),
\end{equation}
where the feedforward connection matrix is a random matrix with $C_{ij}^{a0}=1$,
with probability $P_{ij}^{a0}$ and 0 otherwise. On average, neurons receive
input from $K_0=cK$ input neurons and the connection probability,
$P_{ij}^{a0}$ varies with the difference in position between the pre- and
post synaptic cell as a periodic Gaussian with period $2\pi$ and variance
$\sigma_{a0}$, 
$P_{ij}^{a0}=\frac{2\pi K_0}{N_0}G(\phi_i^a-\phi_j^0,\sigma_{a0})$, 
where 
\begin{equation}
G(\theta,\sigma)\equiv\frac{1}{\sqrt{2\pi}\sigma}
\sum_{k=-\infty}^{\infty}\exp\left(-\frac{(\theta+2k\pi)^2}{2\sigma^2}\right).
\end{equation}

The width $\sigma_{a0}$ determines the spread of preferred feature values of
the input neuron that project to a particular cell. Since 
$0\leq P_{ij}^{a0}\leq 1$, $\sigma_{a0}$ is bounded by a minimal value 
$\sigma_{a,min}$\@. However, we will consider the network in the sparse limit
where $K/N_0\rightarrow 0$ (see below). In this limit 
$\sigma_{a,min}\rightarrow 0$\@. Thus we have two extreme cases: The case where 
$\sigma_{a0}\rightarrow\infty$, in which neuron $(i,a)$ receives input from,
on average, $K_0$ excitatory input cells whose orientations are
chosen randomly, and the case where $\sigma_{a0}=0$, in which neuron $(i,a)$
receives feedforward input from, on average, $K_0$ cells, all of which have 
preferred feature value $\phi_i^a$\@. In the latter case the total feedforward 
input into neuron $(i,a)$ is as strongly modulated with the feature value as 
the output of cells in the input layer and for neighboring cells in the network
the preferred feature of the total feedforward input is similar. In the first 
case, $\sigma_{a0}\rightarrow\infty$, the 
modulation of the total feedforward input will be of order $1/\sqrt{K_0}$
smaller relative to the mean as is the case for the response of the 
cells in the input layer. Furthermore the preferred feature of the total 
feedforward input into neighboring cells is totally uncorrelated. Thus here 
the probability of recurrent connections between cells in the network is not 
correlated with the preferred feature of the feedforward input. The network 
has no functional feature map. 

We study this model in the limit where the connectivity is sparse and $K$ is
large: We first take the limit $N_a\rightarrow\infty$ for $a=0,E,I$ and then 
the limit $K\rightarrow \infty$\@. 
