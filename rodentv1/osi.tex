\newcommand{\mZero}{m_E^{(0)}}
\newcommand{\mOne}{m_E^{(1)}}
\newcommand{\zZero}{z^{(0)}}
\newcommand{\zOne}{z^{(1)}}
OSI $(=1 - \CircVar)$ is commonly used to quantify the degree of response selectivity of neurons to external stimuli. The OSI for the $i^{th}$ neuron given its tuning curve $m_A^{i}(\theta)$ is defined as: 
\begin{equation}
\mathtt{OSI}_{i} = \frac{| \zOne_i |}{ \zZero_i}, \,\,\,\, |z| = \sqrt{(\mathrm{Re}(z))^2 + (\mathrm{Im}(z))^2}
% s_{i} = \frac{ \zOne_i }{ \zZero_i}, \,\,\,\, z = \sqrt{(\mathrm{Re}(z))^2 + (\mathrm{Im}(z))^2}
\label{defosi}
\end{equation}

where,\\
\begin{equation}
z_i^{(n)} = \frac{1}{\pi} \int_0^{\pi} \! \mathtt{d} \theta \,  m_A^i(\theta) \, \exp (2 \, n \,  j \, \theta)) \,\,\,\,; \,\,\,\, j = \sqrt{-1}
\end{equation}
Having determined all the order parameters, we can now study the effect of $\kappa$ on the degree of OS of the population. We generate tuning as described in Appendix \ref{appendix1} curves and subsequently the OSI distribution. 

%%% Local Variables:
%%% mode: latex
%%% TeX-master: "main"
%%% End:
