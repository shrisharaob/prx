Orientation selectivity (OS) is an exhaustively studied property of the visual cortex. Here we analytically study the generation of OS in a strongly connected random network of binary neurons. The mechanism we propose for the generation of OS emphasizes the role of strong recurrent connections in cortical dynamics. We then extend this model to include a small number of specific connections which depend on the preferred orientation such that the ratio of specific to non-specific connections approaches zero in the limit of infinite network size. Depending on the amount of specific connections, we show that strong OS can emerge through a spontaneous symmetric breaking. Using simulations, we show that several experimental observations can be explained. 

%%% Local Variables:
%%% mode: latex
%%% TeX-master: "main"
%%% End:
