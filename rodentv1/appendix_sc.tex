\subsection{Symmetry breaking: Bump solution} 

\subsubsection{When  $m_0^{(1)} = 0$}

Mean input, \\
\begin{eqnarray}
u_A(\phi) :=&& \lim_{N \rightarrow \infty} \PopAvgSqr{\PopAvg{u_A^i(\phi, t)}{t}}{\delta \phi}\nonumber \\
=&& u_A^{(0)} + u_A^{(1)} \cos(2 \phi)  \\
u_A^{(0)} =&& \sqrt{K} \left[c_{FF} J_{A0} m_0^{(0)} + \sum_B J_{AB} m_B^{(0)} \right] - 1\label{ue0equation} \\
u_E^{(1)} =&& p \; J_{EE}\; m_E^{(1)} \label{ue1} \\
u_I^{(1)} =&& 0 \label{ui1}
\end{eqnarray}

Variance of the input,\\
\begin{eqnarray}
\alpha_A :=&& \lim_{N \rightarrow \infty} \PopAvgDPhi{\PopAvg{(u_A^i(\phi, t))^2}{t}} - \PopAvgDPhi{\PopAvg{(u_A^i(\phi, t))}{t}}^2\nonumber  \\
=&& c_{FF} J_{A0}^2 m_0^{(0)} + \sum_B J_{AB}^2 m_B^{(0)} \label{alphaE}
\end{eqnarray}

Evolution of mean rates is given by,\\
\begin{equation}
\tau_A \frac{d}{dt} m_A(\phi, t) = -m_A(\phi, t) + H\left( \frac{-u_A(\phi, t)}{\sqrt{\alpha_A}} \right)
\end{equation}
\\

At the fixed point the mean activity is completely described by, \\
% \begin{eqnarray}
% m_E^{(0)} =&& \frac{1}{\pi} \int_0^\pi d\phi \; H\left( \frac{1 - u_E^{(0)} - u_E^{(1)} \cos(2 \phi)}{\sqrt{\alpha_A}} \right) \label{me0}\\
% m_E^{(1)} =&& \frac{2}{\pi} \int_0^\pi d\phi \; H\left( \frac{1 - u_E^{(0)} - u_E^{(1)} \cos(2 \phi)}{\sqrt{\alpha_A}}  \right) \cos(2 \phi) \label{me1}
% \end{eqnarray}
% \\
\begin{eqnarray}
m_E^{(0)} =&& \frac{1}{\pi} \int_0^\pi d\phi \; H\left( \frac{-u_E(\phi)}{\sqrt{\alpha_A}} \right) \label{me0}\\
m_E^{(1)} =&& \frac{2}{\pi} \int_0^\pi d\phi \; H\left( \frac{-u_E(\phi)}{\sqrt{\alpha_A}}  \right) \cos(2 \phi) \label{me1}
\end{eqnarray}
\\


$m_E^{(0)}$ and $m_I^{(0)}$ are given by requiring balance in eq. [\ref{ue0equation}], $m_I^{(1)} = 0$, because Eq.[\ref{ui1}]. \\

\subsubsection{Behavior of $m_E^{(0)}$ and $m_E^{(1)}$ as $p$ is varied}
In the limit $K \rightarrow \infty$ limit, $m_E^{(0)}$ will remain unchanged as $p$ is varied.\\

Expanding $H(\cdot)$ at the fixed poing of Eq. \ref{me1} for small $m_E^{(1)}$ and integrating,


\begin{eqnarray}
m_E^{(1)} \, F(p) + \left( m_E^{(1)} \right)^3 \,  G(p) + \mathcal{O} \left(\left( m_E^{(1)} \right)^5 \right) = 0
\end{eqnarray}

If, $\exists p_c: F(p_{c}) = 0$, then $\forall p>p_c$, $m_E^{(1)} \neq 0$ solutions are stable if $\sign (G(p_c)) = -1 $  

\begin{eqnarray}
%p_c = 3.404
p_{c} &= \frac{ - \sqrt{\alpha_E^{\star (0)}}}{J_{EE} H^{\prime}(h^{\star})}; \,\,\,\, h^{\star} = \frac{- u_E^{\star (0)}}{\sqrt{\alpha_E^{\star (0)}}} 
\end{eqnarray}

\subsubsection{Stability of the bump solution}
When $K \rightarrow \infty$, 
\begin{eqnarray}
G(p_c)  =&& -\frac{p^3_c J_{EE}^{3}}{\left( \alpha_E^(0) \right)^{\frac{3}{2}}} \left[ \frac{1}{8} H^{\prime \prime \prime} (A_0) - \frac{1}{4} \frac{\left( H^{\prime \prime} (A_0)\right)^2 }{H^{\prime}(A_0)} \right] \\
=&& - \frac{e^{\frac{-A^2_0}{2}}}{\sqrt{2 \pi}} \frac{p^3_c J_{EE}^{3}}{\left( \alpha_E^(0) \right)^{\frac{3}{2}}}  \left[1 + A^2_0 \right] \\
\Rightarrow&& \sign(G(p_c)) = -1
\end{eqnarray}
where, $A_{0} = H^{-1}(m_E^{(0)}))$

\subsubsection{Dynamics of the phase}

\begin{itemize}
  \item The final position of the bump depends on the initial conditions
  \item For a finite network, there is a discrete number of fixed points. The drift term approaches zero as $N \rightarrow \infty$
\end{itemize}



\subsubsection{When  $\gamma, m_0^{(1)} \neq 0$} 

\paragraph{OSI distribution}
\newcommand{\mZero}{m_E^{(0)}}
\newcommand{\mOne}{m_E^{(1)}}
\newcommand{\zZero}{z^{(0)}}
\newcommand{\zOne}{z^{(1)}}
OSI $(=1 - \CircVar)$ is commonly used to quantify the degree of response selectivity of neurons to external stimuli. The OSI for the $i^{th}$ neuron given its tuning curve $m_A^{i}(\theta)$ is defined as: 
\begin{equation}
\mathtt{OSI}_{i} = \frac{| \zOne_i |}{ \zZero_i}, \,\,\,\, |z| = \sqrt{(\mathrm{Re}(z))^2 + (\mathrm{Im}(z))^2}
% s_{i} = \frac{ \zOne_i }{ \zZero_i}, \,\,\,\, z = \sqrt{(\mathrm{Re}(z))^2 + (\mathrm{Im}(z))^2}
\label{defosi}
\end{equation}

where,\\
\begin{equation}
z_i^{(n)} = \frac{1}{\pi} \int_0^{\pi} \! \mathtt{d} \theta \,  m_A^i(\theta) \, \exp (2 \, n \,  j \, \theta)) \,\,\,\,; \,\,\,\, j = \sqrt{-1}
\end{equation}
Having determined all the order parameters, we can now study the effect of $\kappa$ on the degree of OS of the population. We generate tuning as described in Appendix \ref{appendix1} curves and subsequently the OSI distribution. 

%%% Local Variables:
%%% mode: latex
%%% TeX-master: "main"
%%% End:


\paragraph{Virtual rotation}




%%% Local Variables:
%%% mode: latex
%%% TeX-master: "main"
%%% End:
