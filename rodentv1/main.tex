% ****** Start of file apssamp.tex ******
%
%   This file is part of the APS files in the REVTeX 4.1 distribution.
%   Version 4.1r of REVTeX, August 2010
%
%   Copyright (c) 2009, 2010 The American Physical Society.
%
%   See the REVTeX 4 README file for restrictions and more information.
%
% TeX'ing this file requires that you have AMS-LaTeX 2.0 installed
% as well as the rest of the prerequisites for REVTeX 4.1
%
% See the REVTeX 4 README file
% It also requires running BibTeX. The commands are as follows:
%
%  1)  latex apssamp.tex
%  2)  bibtex apssamp
%  3)  latex apssamp.tex
%  4)  latex apssamp.tex
%
\documentclass[%
 reprint,
%superscriptaddress,
%groupedaddress,
%unsortedaddress,
%runinaddress,
%frontmatterverbose, 
%preprint,
%showpacs,preprintnumbers,
%nofootinbib,
%nobibnotes,
%bibnotes,
 amsmath,amssymb,
 aps,
%pra,
%prb,
%rmp,
%prstab,
%prstper,
%floatfix,
]{revtex4-1}

\usepackage{graphicx}% Include figure files
\usepackage{dcolumn}% Align table columns on decimal point
\usepackage{bm}% bold math
%\usepackage{hyperref}% add hypertext capabilities
%\usepackage[mathlines]{lineno}% Enable numbering of text and display math
%\linenumbers\relax % Commence numbering lines

%\usepackage[showframe,%Uncomment any one of the following lines to test 
%%scale=0.7, marginratio={1:1, 2:3}, ignoreall,% default settings
%%text={7in,10in},centering,
%%margin=1.5in,
%%total={6.5in,8.75in}, top=1.2in, left=0.9in, includefoot,
%%height=10in,a5paper,hmargin={3cm,0.8in},
%]{geometry}

\usepackage{lipsum}


\newcommand{\CircVar}{\text{CircVar}}
\newcommand{\Avg}[1]{\left\langle{#1}\right\rangle}
\newcommand{\PopAvg}[2]{\left\langle{#1}\right\rangle_{#2}}
\newcommand{\PopAvgSqr}[2]{\left[{#1}\right]_{#2}}
\newcommand{\PopAvgDPhi}[1]{\left[{#1}\right]_{\delta \phi}}
\newcommand{\Hb}{\left(\frac{1 - u_E^{(0)}}{\sqrt{\alpha_E}}\right)}
\newcommand{\Hx}{\frac{\kappa J_{EE} }{\sqrt{\alpha_E}}}
\newcommand{\Hz}{\frac{u_E^{(1)}}{\sqrt{\alpha_E}}}
\newcommand{\Hzmone}{\frac{\kappa J_{EE} m_E^{(1)}}{\sqrt{\alpha_E}}}
\newcommand{\CZero}{\frac{ exp \left[ \frac{-1}{2} \Hb^2 \right]}{\sqrt{2 \pi}}}
\newcommand{\COne}{\frac{1}{2} \Hb \CZero}
\newcommand{\CTwo}{\frac{1}{6} \CZero \left[ 1 - \Hb^2 \right]}

\newcommand{\sign}{\text{sign}}

\def\defeql{=\mathrel{\mathop:}}

%\newcommand{\costh}{\cos 2 \theta}

\newcommand{\uEOne}{u_E^{(1)}}
\newcommand{\uOne}{\kappa J_{EE} m_E^{(1)}}
\newcommand{\uEZero}{u_E^{(0)}}
\newcommand{\mEOne}{m_E^{(1)}}
\newcommand{\mEZero}{m_E^{(0)}}
\newcommand{\mIZero}{m_I^{(0)}}
\newcommand{\costh}{ \cos 2 \theta}
\newcommand{\Hp}{H^{\prime}(A_0)}
\newcommand{\Hpp}{H^{\prime \prime}(A_0)}
\newcommand{\Hppp}{H^{\prime \prime \prime}(A_0)}

%\newcommand{\PopAvg}[2]{\left\langle{#1}\right\rangle_{#2}}
%\newcommand{\PopAvgSqr}[2]{\left[{#1}\right]_{#2}}
%\newcommand{\PopAvgDPhi}[1]{\left[{#1}\right]_{\delta \phi}}

%%% Local Variables:
%%% mode: latex
%%% TeX-master: "main"
%%% End:



\begin{document}

\preprint{APS/123-QED}

\title{Theory of orientation selectivity in random binary networks} 
\thanks{A footnote to the article title}%

\author{Ann Author}
 \altaffiliation[Also at ]{Physics Department, XYZ University.}%Lines break automatically or can be forced with \\
\author{Second Author}%
 \email{Second.Author@institution.edu}
\affiliation{%
 Authors' institution and/or address\\
 This line break forced with \textbackslash\textbackslash
}%

\collaboration{MUSO Collaboration}%\noaffiliation

\author{Charlie Author}
 \homepage{http://www.Second.institution.edu/~Charlie.Author}
\affiliation{
 Second institution and/or address\\
 This line break forced% with \\
}%
\affiliation{
 Third institution, the second for Charlie Author
}%
\author{Delta Author}
\affiliation{%
 Authors' institution and/or address\\
 This line break forced with \textbackslash\textbackslash
}%

\collaboration{CLEO Collaboration}%\noaffiliation

\date{\today}% It is always \today, today,
             %  but any date may be explicitly specified

\begin{abstract}
Orientation selectivity (OS) is an exhaustively studied property of the visual cortex. Here we analytically study the generation of OS in a strongly connected random network of simple binary neurons. The mechanism we propose for the generation of OS emphasizes the role of strong recurrent connections in cortical dynamics. We then extend this model to include a small number of specific connections which depend on the preferred orientation such that the ratio of specific to non-specific connections approaches zero in the appropriate limit. Depending on the amount of specific connections, we show that strong OS can emerge through a spontaneous symmetric breaking. Using simulations, we show that several experimental observations can be explained. 

\begin{description}
\item[Usage]
Secondary publications and information retrieval purposes.
\item[PACS numbers]
May be entered using the \verb+\pacs{#1}+ command.
\item[Structure]
You may use the \texttt{description} environment to structure your abstract;
use the optional argument of the \verb+\item+ command to give the category of each item. 
\end{description}
\end{abstract}

\pacs{Valid PACS appear here}% PACS, the Physics and Astronomy
                             % Classification Scheme.
%\keywords{Suggested keywords}%Use showkeys class option if keyword
                              %display desired
\maketitle

\tableofcontents  

\section{\label{sec:level1}Introduction}

Orientation selectivty has been experimentally observed in the neurons of the primary visual cortex (V1) of various species. The cortical organization of neurons in some of these species like the primates exhibit orientation maps. In the theories prevalent of OS, such an arrangement of neurons is often associated with functional connectivity, i.e. neurons with similar preferred orientation have a higher probability of connections. The necessity of such a connectivity scheme has seldom been questioned\@. 
Carl van Vreeswijk and Haim Somplolinsky \cite{} showed that  under very general conditions, a random network of binary neurons with excitatory and inhibitory synapses automatically converges to a fixed point where the total excitatory and inhibitory currents cancel each other on average. This fixed point is the balanced state. In this state, both the mean and fluctuation of the total input is of the order of the threshold\@. \\
In section \ref{sec:level2}, we analytically demonstrate that OS can emerge in a random network of binary neurons without any functional connectivity, provided the network is operating in the balanced regime. Experimental data from the rodents provides further support to our theory. Alghough most members of the rodent family(order) lack orientation maps, strong OS is observed in the V1 of these animals. A previous study demonstrated this mechanism through numerical simulations of a network of conductance based neurons. \cite{}\@. 
Experimental results indicate significant changes occuring in the selectivity of rodent V1 neurons during the critical period(P23 - 30). After maturation, cortical synaptic weights exhibit long-tailed distributions with only a small number of strong synapses. In section \ref{sec:specific_con}, we will study the effects of introducing a very small number of funcionally specific connections in a ring model operating in the balanced regime. Under certain conditions as shall be described, this leads to spontaneous symmetry breaking. Finally, in section \ref{sec:simulations}, we will present the results of finite N simulations and compare it to the theory. 
 
%%% Local Variables:
%%% mode: latex
%%% TeX-master: "main"
%%% End:


\section{\label{sec:level2}Binary model}
%\lipsum[2-4]

We model a network of neurons that are selective for a 1-dimensional stimulus 
feature $\theta$\@. To avoid the difficulty of having to deal with what happens
at the boundary we assume that $\theta$ is a periodic  variable with period 
$2\pi$\@.. 
The network consists of $N_E$ excitatory and $N_I$ inhibitory 
neurons organized on a $2\pi$ periodic ring. We denote neuron $i$ of 
population $a$ (with $a=E,I$ and $i=1,2,\ldots,N_a$) as neuron $(i,a)$. 
The position on the ring of neuron $(i,a)$ is $\phi_i^a=2\pi i/N_a$.

The neurons are modeled as sequentially updated
binary units. The state, $\sigma_i^a$, of neuron $(i,a)$ is updated, 
on average, once per time constant $\tau_a$\@. It is set to zero if, at the 
time of the update, its input, $u_i^a$, is below the threshold, $T_a$, 
and set to one otherwise.

The input, $u_i^a$, has three components: the feedforward input, $u_i^{a0}$, 
from excitatory units in the input layer, the recurrent excitatory input, 
$u_i^{aE}$, and the recurrent inhibitory feedback, $u_i^{aI}$\@. 

The recurrent excitation and inhibition are given by
\begin{equation}
u_i^{ab}(t)=\frac{J_{ab}}{\sqrt{K}}\sum_{j=1}^{N_b}C_{ij}^{ab}\sigma_j^b(t)
\qquad \qquad (b=E,I),
\end{equation}
where $J_{ab}/\sqrt{K}$, with $J_{aE}>0$ and $J_{aI}<0$, is the contribution 
of an active presynaptic cell to the recurrent input.
The connection matrix $C_{ij}^{ab}$ is randomly chosen: $C_{ij}^{ab}=1$ 
with probability $P_{ij}^{ab}=\frac{K}{N_b}[1+2p_{ab}\cos (\phi_i^a-\phi_j^b)]$
and $C_{ij}^{ab}=0$ otherwise. Thus, on average a neuron receives recurrent 
input from $K$ excitatory and $K$ inhibitory units. The modulation, $p_{ab}$, 
$0\leq p_{ab} \leq 1/2$, determines how much the connection probability from 
population $b$ to population $a$ depends on the distance between neurons on 
the ring. Note that, for convenience in the analysis we have assumed that the 
distance dependence of the connection probability is given by 
$1+p\cos(\Delta \phi)$. However, other unimodal probability profiles that are 
symmetric and have their peak at $\Delta\phi=0$ show qualitatively similar 
behavior.

The neurons in the network receive feedforward input, $u_i^{a0}$, from 
excitatory neurons in the input layer.
We assume that there is a pool of $N_0$ of such excitatory neurons 
whose average activity, $m_i^0$, is given by
\begin{equation}
m_i^0=m_0[1+\mu\cos (\theta-\phi_i^0)]\equiv m_0(\theta-\phi_i^0).
\end{equation}.
Here $\theta$ is the feature value of the stimulus and $\phi_i^0=2\pi i/N_0$ 
is the preferred feature value of neuron $(i,0)$\@. The average activity, 
$m_0$, is an increasing function of the stimulus intensity and $\mu$ determines 
how much the response of the input cells is modulated with the stimulus feature.

We consider two models for the pool of input units. In the first the activity,
$\sigma_u^0$ is a continous variable that does not change over time, 
$\sigma_i^a(t)=m_i^a$\@. In the second model the input units are binary 
variable that are sequentially updated, with each cell updated on average once 
per time constant $\tau_0$, at the up[date $\sigma_i^0$ is set to 1 with 
probability $m_i^0$ and set to zero otherwise.

The feedforward input, $u_i^{a0}$, is given by
\begin{equation}
u_i^{a0}(t)=\frac{J_{a0}}{c\sqrt{K}}\sum_{j=1}^{N_0}C_{ij}^{a0}\sigma_j^0(t),
\end{equation}
where the feedforward connection matrix is a random matrix with $C_{ij}^{a0}=1$,
with probability $P_{ij}^{a0}$ and 0 otherwise. On average, neurons receive
input from $K_0=cK$ input neurons and the connection probability,
$P_{ij}^{a0}$ varies with the difference in position between the pre- and
post synaptic cell as a periodic Gaussian with period $2\pi$ and variance
$\sigma_{a0}$, 
$P_{ij}^{a0}=\frac{2\pi K_0}{N_0}G(\phi_i^a-\phi_j^0,\sigma_{a0})$, 
where 
\begin{equation}
G(\theta,\sigma)\equiv\frac{1}{\sqrt{2\pi}\sigma}
\sum_{k=-\infty}^{\infty}\exp\left(-\frac{(\theta+2k\pi)^2}{2\sigma^2}\right).
\end{equation}

The width $\sigma_{a0}$ determines the spread of preferred feature values of
the input neuron that project to a particular cell. Since 
$0\leq P_{ij}^{a0}\leq 1$, $\sigma_{a0}$ is bounded by a minimal value 
$\sigma_{a,min}$\@. However, we will consider the network in the sparse limit
where $K/N_0\rightarrow 0$ (see below). In this limit 
$\sigma_{a,min}\rightarrow 0$\@. Thus we have two extreme cases: The case where 
$\sigma_{a0}\rightarrow\infty$, in which neuron $(i,a)$ receives input from,
on average, $K_0$ excitatory input cells whose orientations are
chosen randomly, and the case where $\sigma_{a0}=0$, in which neuron $(i,a)$
receives feedforward input from, on average, $K_0$ cells, all of which have 
preferred feature value $\phi_i^a$\@. In the latter case the total feedforward 
input into neuron $(i,a)$ is as strongly modulated with the feature value as 
the output of cells in the input layer and for neighboring cells in the network
the preferred feature of the total feedforward input is similar. In the first 
case, $\sigma_{a0}\rightarrow\infty$, the 
modulation of the total feedforward input will be of order $1/\sqrt{K_0}$
smaller relative to the mean as is the case for the response of the 
cells in the input layer. Furthermore the preferred feature of the total 
feedforward input into neighboring cells is totally uncorrelated. Thus here 
the probability of recurrent connections between cells in the network is not 
correlated with the preferred feature of the feedforward input. The network 
has no functional feature map. 

We study this model in the limit where the connectivity is sparse and $K$ is
large: We first take the limit $N_a\rightarrow\infty$ for $a=0,E,I$ and then 
the limit $K\rightarrow \infty$\@. 

The analysis is similar to that of van Vreeswijk and Sompolinsky 
\cite{vanVreeswijk2005}, to which we refer for further details.
 
\subsection{Coarse grained analysis}
Because the strength of the synapses scales as $1/\sqrt{K}$, temporal 
fluctuations in the input do not vanish in the large $K$ limit. 
Furthermore, because  of the randomness of the connectivity and feedforward 
input, the time-average input for neurons of the same population is not the 
same, even for neurons which are very close. 
 
To analyze the network response we calculate the statistics of the different
variables for neurons $(i,A)$ with $\phi_i^a$ between $\phi$ and
$\phi+\delta\phi$ for a sufficiently small $\delta\phi$\@. Sincw we consider
the network in the limit $N_a\rightarrow \infty$, we can make $\delta\phi$ 
arbitrarily small.
Averages over this subpopulation of variables $X_i^a$ are denoted by
$\langle X_i^a\rangle_{\phi,\delta\phi}$\@. Because of the 
rotational and mirror symmetry of the system we have that, for a stimulus with 
orientation $\theta$, this average satisfies 
$\langle u_i^A(t)\rangle_{\phi,\delta\phi}=u_A(\theta-\phi,t)$, where $u_A$ is 
symmetric. In fact for all variables $X$ of interest the average will satisfy $\langle X_i^A \rangle_{\phi,\delta \phi}=f_A
(\theta-\phi)$ for some symmetric function $f_A$. For exmple, when the network has reached its equilibium, we have that
$\langle\sigma_i^a(t)\rangle_{\phi,\delta\phi}=
\langle m_i^a\rangle_{\phi,\delta\phi}=m_a(\theta-\phi)$\@. Here $m_i^a$ is the
the time averaged value of $\sigma_i^a$\@. Because of this rotational symmetry,
it is sufficient to calculate the average quantities only for $\phi=0$ at 
different values $\theta$ of the stimulus. For notational convenience we will 
denote $\langle \cdot \rangle_{0,\delta \phi}$ by 
$\langle \cdot \rangle_{\delta \phi}$\@.

In the steady state, when the activities are constant, we can write the input 
into neuron $(i,a)$, for a stimulus with orientation $\theta+\phi_i^a$ as
\begin{equation}
u_i^a(t)=u_{a}(\theta)+\Delta u_i^{a}(\theta)+\delta u_i^{a}(t),
\label{input_terms:eq}
\end{equation}
where $u_{a}$ is the population averaged input and the quenched disorder in 
$u_i^a$, $\Delta u_i^{a}$, is the
difference between the averaged input into the neuron and the population 
average, $\Delta u_i^{a}\equiv \langle u_i^a-u_a\rangle_t$, where
$\langle\cdots\rangle_t$ denotes the average over time.
Finally, $\delta u_i^{a}(t)$ represents the temporal fluctuation in the input.

Because $\Delta u_i^a$ and $\delta u_i^{a}(t)$ are composed of many small
contributions, they have Gaussian statistics.
Thus we alternatively can write
\begin{equation}
u_i^a(t)=u_{a}(\theta)+\sqrt{\gamma_a(\theta)}x_i^a(\theta)+
\sqrt{\alpha_a(\theta)-\gamma_a(\theta)}y_i^a(t),
\label{input_terms1:eq}
\end{equation}
where $\gamma_a$ and $\alpha_a-\gamma_a$ are the variance of the quenched 
disorder and the temporal fluctuations respectively, and $x_i^a$ and $y_i^a$ 
are Gaussian random variables with mean 0 and variance 
$\langle [x_i^a]^2\rangle_{\delta \phi}=
\langle [y_i^a(t)]^2\rangle_{\delta \phi}=1$\@. Furthermore, for sufficiently 
large $t-t^\prime$, $y_i^a(t)$ and $y_i^a(t^\prime)$ are uncorrelated,
$\langle y_i^a(t)y_i^a(t^\prime)\rangle_{\delta \phi}\rightarrow 0$, for
$|t-t^\prime|\rightarrow\infty$\@.

In the steady state, the average activity of neuron $(i,a)$, $m_i^a=
\langle \sigma_i^a\rangle_t$, is equal to the probability that the
fluctuations bring the input above threshold,
\begin{equation}
m_i^a=\int\! Dy
\,\Theta \left(u_a+\sqrt{\gamma_a}x_i^a+\sqrt{\alpha_a-\gamma_a}y-T_a\right),
\end{equation}
where $\Theta$ is the Heaviside function, $\Theta(x)=1$ for $x>0$ and
$\Theta(x)=0$ for $x<0$and $Dy$ is the Gaussian measure, 
$Dy=exp(-y^2/2)dy/\sqrt{2\pi}$\@. This can be written as
\begin{equation}
m_i^a=H\left(\frac{T_a-u_a-\sqrt{\gamma_a}x_i^a}{ \sqrt{\alpha_a-\gamma_a}}
\right).
\label{indiv-rate:eq}
\end{equation}
Here $H$ denotes the cumulative Gaussian distribution,
$H(x)=\frac{1}{\sqrt{2\pi}}\int_x^\infty\!dy\,e^{-y^2/2}$\@.

Because we know the distribution of $x_i^a$, Eqn.~(\ref{indiv-rate:eq}) 
determines, for a stimulus with given intensity, $m_0$, and feature value, 
$\theta$, the distribution 
of the activities for neurons $(i,a)$ with $\phi<\phi_i^a<\phi+\delta\phi$, 
if we know $u_a$, $\alpha_a$ and $\gamma_a$\@. 

However, we also need to take into account that the random variable $x_i^a$
changes when the stimulus parameters are changed. For example, if
$\theta$ is changed to $\theta^\prime$, $x_i^a$ changes to
$x_i^{\prime a}$ and one would expect $x_i^{\prime a}$ to be close to
$x_i^a$ if $\theta-\theta^\prime$ is small. Thus the input statistics of 
population $a$ is fully known if we have a
description of $u_a$ and $\alpha_a$ for any $m_0$ and $\theta$ and if we know 
the correlation in the quenched disorder for any pair of stimuli,
$(m_0,\theta)$ and $(m_0^\prime,\theta^\prime)$.  Below we will sketch
the analysis to derive these parameters.

A convenient feature of the binary neuron model is that the average activity,
$m_a\equiv\langle m_i^a\rangle_{\delta\phi}$, given by
\begin{eqnarray}
m_a(\theta) & = &\int\! Dx
H\left(\frac{T_a-u_a(\theta)-\sqrt{\gamma_a(\theta)}x}
{ \sqrt{\alpha_a(\theta)-\gamma_a(\theta)}}\right) \nonumber \\
 & = & H\left(\frac{T_a-u_a(\theta)}{\sqrt{\alpha_a(\theta)}}\right),
\label{meanrate:eq}
\end{eqnarray}
is independent of $\gamma_a$\@. Thus if we can express $u_a$ and $\alpha_a$ 
in the mean activities, $m_b$, this together with Eqn.~(\ref{meanrate:eq}) 
determines these quantities.

\subsection{Population averaged inputs}
The term $u_a(\theta)$ in Eqn.~(\ref{input_terms:eq}) is the average 
of the input $u_i^a$, for neurons 
in population $a$ with $0<\phi_i^a<\delta \phi$, when the stimulus feature 
value is $\theta$. It satisfies
$u_a(\theta)=\sum_{a=0,E,I}u_{ab}(\theta)$.
The feedforward input, $u_{a0}$ is given by
\begin{eqnarray}
u_{a0}(\theta) & = & \langle u_i^{a0}\rangle_{\delta\phi}= 
\frac{J_{a0}}{c\sqrt{K}}\sum_{j=1}^{N_0}\langle C_{ij}^{a0}
\rangle_{\delta\phi}m_0(\theta-\phi_j^0)
\nonumber \\
   & = &  \sqrt{K}J_{a0}\int_0^{2\pi} d\phi^\prime\, G(\phi^\prime,\sigma_{a0})
m_0(\theta-\phi^\prime).
\end{eqnarray}
Using $\int_0^{2\pi} d\phi^\prime\, G(\phi^\prime,\sigma)\exp(ni\phi^\prime) =
exp(-n^2\sigma^2/2)$ this can be written as
\begin{equation}
u_{a0}(\theta)=\sqrt{K}J_{a0}m_0[1+\mu\zeta_a\cos \theta],
\label{uA0:eq}
\end{equation}
where $\zeta_a\equiv e^{-\sigma_{a0}^2/2}$\@.

Similarly the feedback input $u_{ab}$, with $b=E,I$, satisfies
\begin{eqnarray}
u_{ab}(\theta) & = & \langle u_i^{ab}\rangle_{\delta\phi} = 
\frac{J_{ab}}{\sqrt{K}} \sum_{j=1}^{N_b}
\langle C_{ij}^{ab}\rangle_{\delta\phi}m_b(\theta-\phi_j^b)
\nonumber \\
& = & \sqrt{K}J_{ab}\left[ m_b^{(0)}+p_{ab}m_b^{(1)}\cos \theta\right].
\label{uAB:eq}
\end{eqnarray}
Here $m_a^{(k)}$ is the $k$th Fourier moment of $m_a$, $m_a(\theta)=
\sum_{k=0}^\infty m_a^{(k)}\cos k\theta$\@.
 
\subsection{Equal Time Fluctuations of the Inputs}
Because the random connection matrices and feedforward inputs 
are independent, the fluctuations in the different components of the 
input are independent so that we can write for $\alpha_a$,
$\alpha_b(\theta)=\sum_{b=0,E,I}\,\alpha_{ab}(\theta)$,
where $\alpha_{ab}(\theta)=
\langle [u_i^{ab}(t)-u_{ab}(\theta)]^2\rangle_{\delta\phi}$\@.

For the variance of the feedback, $\alpha_{ab}$, with $b=E,I$, we have
(see \cite{vanVreeswijk2005})
\begin{eqnarray} 
\alpha_{ab}(\theta) & = &\frac{J_{ab}^2}{K}\sum_j
\langle C^{ab}_{ij}\rangle_{\delta\phi} m_b(\theta-\phi_j^B)
\nonumber \\
 & = & J_{ab}^2[m_b^{(0)}+p_{AB}m_b^{(1)}\cos \theta].
\label{alAB:eq}
\end{eqnarray}

The equal time variance of the feedforward input depends on the model of the 
of the external input. If the input is from unit with continuous output,
$\sigma_i^0=m_i^0$, $\alpha_{a0}$ is given by
\begin{eqnarray}  
\alpha_{a0}(\theta) & = &\frac{J_{a0}^2}{c^2K}\sum_j \langle C_{ij}^{a0}
\rangle_{\delta\phi}m_0^2(\theta-\phi_i^0) \nonumber \\
 & = & \frac{[J_{a0}m_0]^2}{c}\Big[1+2\mu\zeta_a\cos \theta+\nonumber \\
 & & \mbox{}+\frac{\mu^2}{2}(1+\zeta_a^4\cos 2\theta)\Big].
\label{alA0:eq}
\end{eqnarray}
Alternatively when the input neurons are modeled as 
binary units with  are randomly updated and set to 1 with a probability
$m_i^0$ and to 0 otherwise, $\alpha_{a0}$ is given by
\begin{eqnarray}  
\alpha_{a0}(\theta) & = &\frac{J_{a0}^2}{c^2K}\sum_j \langle C_{ij}^{a0}
\rangle_{\delta\phi}m_0(\theta-\phi_i^0) \nonumber \\
 & = & \frac{J_{a0}^2m_0}{c}\Big[1+\mu\zeta_a\cos \theta\Big].
\label{alA0a:eq}
\end{eqnarray} 


\subsection{The Balanced Solution}
In the steady state the averaged activities $m_a$ are given by Eqn.~(\ref
{meanrate:eq}), where by combining Eqns.~(\ref{uA0:eq}) and (\ref{uAB:eq}), we 
can write $u_a$ as $u_a(\theta)= u_a^{(0)}+u_a^{(1)}\cos \theta$, while with
Eqns.~(\ref{alAB:eq}) and (\ref{alA0:eq}) we can write $\alpha_a$ as 
$\alpha_a(\theta)= \alpha_a^{(0)}+\alpha_a^{(1)}\cos \theta+
\alpha_a^{(2)}\cos 2\theta$\@.
The equal time fluctuations $\alpha_a$ are of order 1, while $u_a$ is of order 
$\sqrt{K}$, unless the leading terms in $u_a^{(0)}$ and $u_a^{(1)}$ cancel.
In the large $K$ limit this means that, when this cancellation does not take 
place, $m_a$ goes to either $m_a=0$ or $m_a=1$,
depending on the sign of $u_a$. 
Therefore the only way in which the system can have low, but non-zero, 
activities is if in the leading order of both $u_a^{(0)}$ and $u_a^{(1)}$, the 
recurrent inhibitory input cancels the total feedforward and recurrent 
excitatory input.

Imposing this requirement for both populations leads to the balanced
solution where, up to a correction term of order $1/\sqrt{K}$, $m_a$ satisfies
$m_a^{(0)}=A_a^{(0)}m_0$, and $m_a^{(1)}=A_a^{(1)} m_0$, where
\begin{eqnarray}
A_E^{(1)} & = & \frac{J_{EI}J_{I)}-J_{II}J_{E0}}{J_{EE}J_{II}-J_{EI}J_{IE}},
\nonumber \\
A_I^{(1)} & = & \frac{J_{IE}J_{E0}-J_{EE}J_{I0}}{J_{EE}J_{II}-J_{EI}J_{IE}},
\nonumber \\
A_E^{(1)} & = & \mu\frac{p_{EI}\zeta_I J_{EI}J_{I0}-p_{II}\zeta_E J_{II}J_{E0}}
             {p_{EE}p_{II}J_{EE}J_{II}-p_{EI}p_{IE}J_{EI}J_{IE}},
\nonumber \\
A_I^{(1)} & = & \mu\frac{p_{IE}\zeta_E J_{IE}J_{E0}-p_{EE}\zeta_I J_{EE}J_{I0}}
             {p_{EE}p_{II}J_{EE}J_{II}-p_{EI}p_{IE}J_{EI}J_{IE}}.
\end{eqnarray}
Since $m_a(\theta)$ need to be positive, The connection strengths 
$J_{ab}$ need to be chosen such that $A_a^{(0)}$ is positive and the tuning 
of the feedback connections, $p_{ab}$, need to be chosen such that
$|A_a^{(1)}|<A_a^{(0)}/\mu\zeta_a$\@. If we further impose that $|J_{EE}J_{II}|<
|J_{EI}J_{IE}|$ it can be shown that the network will always evolve to the 
balanced state, provided that the average update time of the inhibitory cells, 
$\tau_I$, is sufficiently small compared to $\tau)E$, the update time of the 
excitatory population \cite{vanVreeswijk1998}.

The balance equatios determines $\alpha_a$ in the large $K$ limit (Eqn.~(\ref{alAB:eq})). 
But $u_a$ now depends on 
the $1/\sqrt{K}$ corrections of the activity and remains to be determined. 
The average, $u_a^{(0)}$, and modulation, $u_a^{(1)}$, are determined by
\begin{equation}
m_a^{(0)}=\frac{1}{2\pi}\int_0^{2\pi}\!d\theta\,
H\left(\frac{T_A-u_a^{(0)}-u_a^{(1)}\cos \theta}{\sqrt{\alpha_a(\theta)}}
\right)
\end{equation}
and
\begin{equation}
m_a^{(1)}=\frac{1}{\pi}\int_0^{2\pi}\!d\theta\,
H\left(\frac{T_A-u_a^{(0)}-u_a^{(1)}\cos \theta}{\sqrt{\alpha_a(\theta)}}
\right)\cos \theta.
\end{equation}

\subsection{Statistics of the Quenched Disorder:}
The activity $m_i^a(\theta)$ satisfies
\begin{equation}
m_i^a(\theta)=H\left(\frac{T_A-u_a(\theta)-\sqrt{\gamma_a(\theta)}
x_i^a(\theta)}{\sqrt{\alpha_a(\theta)-\gamma_a(\theta)}}\right).
\label{ind-rate:eq}
\end{equation}
Since $x_i^a(\theta)$ is drawn from a Gaussian with mean 0 and variance 1,
this completely determines the distribution of activitie, 
$\Pr[m_i^a(\theta)]$, if $\gamma_a$ is known. However, to calculate the joint 
distribution, $\Pr[m_i^a(\theta_1),m_i^a(\theta_2),\ldots,m_i^A(\theta_n)]$ of 
the activities of a neuron, for stimulus feature values 
$\theta_1,\theta_2,\ldots,\theta_n$, we need to know the joint statistics of 
disorder variables, $x_i^a(\theta_1),
x_i^a(\theta_2),\ldots,x_i^a(\theta_n)$\@. Luckily these are Gaussian 
random variables, so that their joint statistics are fully determined by the 
cross-correlations, 
$\langle(x_i^a(\theta_k)x_i^a(\theta_l)\rangle_{\delta\phi}$\@.
We will now determine these correlations.

It is convenient to write the correlations between $x_i^a$ at angle 
$\theta+\Delta$ and $x_i^a$ at angle $\theta-\Delta$ as
$\langle x_i^a(\theta+\Delta)x_i^a(\theta-\Delta)\rangle_{\delta\phi}=
\frac{\beta_a(\theta,\Delta)}
{\sqrt{\gamma_a(\theta+\Delta)\gamma_a(\theta-\Delta)}}$\@. Since
$\langle [x_i^a(\theta)]^2\rangle_{\delta\phi}=1$, we have that 
$\beta_a(\theta,0)=\gamma_a(\theta)$\@.

We introduce a new order parameter, $q_a$ defined by $q_a(\theta,\Delta)\equiv
\langle m_i^a(\theta+\Delta)m_i^a(\theta-\Delta)\rangle_{\delta\phi}$\@.
This is the joint probability of a neuron being in the active state both
for a stimulus at $\theta+\Delta$ and at $\theta-\Delta$.
It can be calculated using Eqn.~(\ref{ind-rate:eq}), by averaging over 
the correlated Gaussian variables $x_i^a(\theta+\Delta)$ and 
$x_i^a(\theta-\Delta)$\@. After some algebra one obtains
\begin{eqnarray}
q_{a}(\theta,\Delta)  & = & 
\int\!Dx\, H\left(\frac{T_a-u_a^+-\sqrt{\beta_a}x}
{\sqrt{\alpha_a^+-\beta_a}}\right)\times \nonumber \\
 & & \makebox[0.3in]{}\times H\left(\frac{T_a-u_a^--\sqrt{\beta_a}x}
{\sqrt{\alpha_a^--\beta_a}}\right),
\label{qA:eq}
\end{eqnarray}
where we used the abbreviations, $u_a^\pm=u_{a}(\theta\pm\Delta)$,
$\alpha_a^\pm=\alpha_{a}(\theta\pm\Delta)$ and
$\beta_a=\beta_{a}(\theta,\Delta)$\@.

Following the same logic as for the fluctuations $\alpha_a$, we can write for
the correlations $\beta_{a}$, $\beta_{a}(\theta,\Delta)=
\sum_{b=0,E,I}\beta_{ab}(\theta,\Delta)$, where 
$\beta_{ab}(\theta,\Delta)=
\langle u_i^{ab}(t)u_i^{ab}(t^\prime)\rangle_{\delta\phi}-
\langle u_i^{ab}(t)\rangle_{\delta\phi}
\langle u_i^{ab}(t^\prime)\rangle_{\delta\phi}$ is the 
contribution to the input correlation due to input from population $b$\@.

The contribution to this correlation from the external input is given by
\begin{eqnarray}  
\beta_{a0}(\theta,\Delta) & = & \frac{J_{a0}^2 m_0^2}{c^2K}\sum_j
\langle C_{ij}^{a0}\rangle_{\delta\phi}\times \nonumber \\
 & & \makebox[0.1in]{}\times m_0(\theta+\Delta-\phi_j^0)
       m_0(\theta-\Delta-\phi_j^0)
\nonumber \\
 & = & \frac{J_{a0}^2 m_0^2}{c}\Big[
1+2\mu\zeta_a\cos\theta\cos\Delta + \nonumber \\
 & & \makebox[0.2in]{}+\frac{\mu^2}{2}(\cos 2\Delta+\zeta_A^4\cos 2\theta)
\Big].
\end{eqnarray}

The input correlations due to the two feedback components, $\beta_{ab}$,
with $b=E,I$, depend on $q_{a}$ and are given by
(see \cite{vanVreeswijk2005})
\begin{eqnarray} 
\beta_{ab}(\theta,\Delta) & = &
\frac{J_{ab}^2}{K}\sum_j\langle C_{ij}^{a0}\rangle_{\delta\phi}
q_b(\theta-\phi_j^b,\Delta) \nonumber \\
 &= &J_{ab}^2[q_{b}^{(0)}(\Delta)+p_{ab}q_{b}^{(1)}(\Delta)\cos\theta],
\label{betaA:eq}
\end{eqnarray}
where $q_{b}^{(k)}$ is the $k$th Fourier component in the variable $\theta$ of
$q_{b}$, $q_{b}(\theta,\Delta)=\sum_k q_{b}^{(k)}(\Delta)\cos k\theta$\@.

This expresses $\beta_{a}(\theta,\Delta)$ in $q_{a}^{(0)}(\Delta)$ and
$q_{a}^{(1)}(\Delta)$\@. Self-consistent solutions for these are obtained 
by imposing $q_{a}^{(0)}(\Delta)=\frac{1}{2\pi}\int_0^{2\pi}\!d\phi\,
q_{a}(\theta,\Delta)$ and $q_{a}^{(1)}(\Delta)=\frac{1}{\pi}
\int_0^{2\pi}\!d\phi\,q_{a}(\theta,\Delta)\cos \theta$ where 
$q_{a}(\theta,\Delta)$ is given by Eqn.~(\ref{qA:eq}).

Extending these results to the case where the feedforward activity, $m_0$, is 
also changed is straightforward:
The population averaged input $u_a$ and input variance $\alpha_a$ are now
functions of $m_0$ and $\theta$, and are calculated as before. 
For the correlations we have to consider 2 stimuli specified by variables
($m_0^+$,$\theta+\Delta$) and ($m_0^-$,$\theta-\Delta$) respectively.  
The correlations in the total input are denoted by 
$\beta_a(m_0^+,m_0^-,\theta,\Delta)$, while for the correlations in the 
activity we write $q_a(m_0^+,m_0^-,\theta,\Delta)$\@.
The corelation $\beta_a$ can be written as
$\beta_a(m_0^+,m_0^-,\theta,\Delta)=\sum_{k=0}^2
\beta_a^{(k)}(m_0^+,m_0^-,\Delta)\cos k\theta$ with
\begin{eqnarray}
\beta_a^{(0)}(m_0^+,m_0^-,\Delta) & = &
\frac{J_{a0}^2}{c}m_0^+m_0^-\left[1+\frac{\mu^2}{2}\cos2\Delta\right]+ 
\nonumber \\
 & & \makebox[0.2in]{}+
\sum_{b=E,I}J_{ab}^2q_b^{(0)}(m_0^+,m_0^-,\Delta) \nonumber \\
\beta_a^{(1)}(m_0^+,m_0^-,\Delta) & = &
\frac{2J_{a0}^2}{c}m_0^+m_0^-\mu\zeta_A\cos\Delta+
\nonumber \\
 & & \makebox[0.2in]{}+
\sum_{b=E,I}p_{ab}J_{ab}^2q_b^{(0)}(m_0^+,m_0^-,\Delta) \nonumber \\
\beta_a^{(0)}(m_0^+,m_0^-,\Delta) & = &
\frac{J_{a0}^2}{2c}m_0^+m_0^-\mu^2\zeta_A^2,
\end{eqnarray}
where 
$q_a^{(k)}(m_0^+,m_0^-,\Delta)$ is the $k$th Fourier moment in $\theta$
of $q_a(m_0^+,m_0^-,\theta,\Delta)$\@. $q_a$ is still given by
Eqn.~(\ref{qA:eq}), except that now $u_a^\pm=u_a(m_0^\pm,\theta\pm\Delta)$,
$\alpha_a^\pm=\alpha_a(m_0^\pm,\theta\pm\Delta)$ and 
$\beta_a=\beta_a(m_0^+,m_0^-,\theta,\Delta)$\@.
A self-consistency requirement equivalent to that given above for 
$m_0^\pm=m_0$  determines $\beta_a$.

\subsection{Symmetries of the Solution}
The connection probabilities are even functions of the difference in 
positions, $\phi_i^a-\phi_j^a$ and the external input is 
symmetric in $\theta-\phi_i^a$\@. This implies that 
$\beta_{a}(m_0,m_0^\prime,\theta,\Delta)=
\beta_{a}(m_0,m_0^\prime,-\theta,-\Delta)$.
As shown above, $\beta_{a}(m_0,m_0^\prime,\theta,\Delta)=
\beta_{a}(m_0,m_0^\prime,-\theta,\Delta)$\@. Together these two symmetries 
also imply that $\beta_{a}(m_0,m_0^\prime,\theta,\Delta)=
\beta_{a}(m_0,m_0^\prime,\theta,-\Delta)$\@. Furthermore, under the 
transformation $(\theta,\Delta)\rightarrow (\theta+\pi,\Delta-\pi)$ the two
input orientations, $\theta_1=\theta+\Delta$ and $\theta_2=\theta-\Delta$, 
transform to $\theta_1 \rightarrow \theta_1$ and 
$\theta_2\rightarrow\theta_2+2\pi=\theta_2$\@. Thus we also have that 
$\beta_{a}(m_0,m_0^\prime,\theta,\Delta)=
\beta_{a}(m_0,m_0^\prime,\theta+\pi,\Delta-\pi)$\@. Finally, if we make the
change, $(m_0,\theta)\rightarrow(m_0^\prime,\theta^\prime)$ and
 
$(m_0^\prime,\theta^\prime) \rightarrow (m_0,\theta)$, the correlations are not 
changed either. Thus $\beta_{a}(m_0,m_0^\prime,\theta,\Delta)=
\beta_{a}(m_0^\prime,m_0,\theta,-\Delta)$\@.

Taking these symmetries into account we can write 
$\beta_{a}(m_0,m_0^\prime,\theta,\Delta)=\beta_{a}^{(0)}
(m_0,m_0^\prime,\Delta)+\beta_{a}^{(1)}(m_0,m_0^\prime,\Delta)\cos \theta$ as
\begin{eqnarray}
\lefteqn{\beta_{a}(m_0,m_0^\prime,\theta,\Delta)= }\nonumber \\
& & \sum_{n=0}^\infty \beta_{a}^{(0,2n)}(m_0,m_0^\prime)\cos 2n \Delta+
\nonumber \\
& & \mbox{}+\sum_{n=0}^\infty 
\beta_{a}^{(1,2n+1)}(m_0,m_0^\prime)\cos \theta\cos (2n+1) \Delta+
\nonumber \\
& & \mbox{}+\beta_{a}^{(2,0)}(m_0,m_0^\prime)\cos 2\theta,
\end{eqnarray}
where $\beta_{a}^{(0,n)}$ and $\beta_{a}^{(1,n)}$ are the $n$th Fourier
components in $\Delta$ of $\beta_{a}^{(0)}$ and $\beta_{a}^{(1)}$
respectively\@. Note that due to the symmetry we also have that
$\beta^{(n,m)}(m_0,m_0^\prime)=
\beta^{(n,m)}(m_0^\prime,m_0)$\@.

\subsection{The solution in the cases without map}
So far we have considered the solution in the general case. What does this
imply for the network without a map ($\zeta_A=0$) and the case with a map?

When there is no map, 
$m_A^{(1)}=0$ and therefore, from Eqn.~(\ref{alAB:eq}) and 
either Eqn.~(\ref{alA0:eq}) or Eqn.~(\ref{alA0a}) that
$\alpha_A^{(1)}=\alpha_a^{(2)}=0$\@. Using Eqn.~(\ref{eq:uA}), one sees that
his implies that $u_A^{(1)}=0$\@. If we now consider Eqns.~(\ref{qA:eq}) and 
(\ref{betaA:eq}) we see that the fact that $u_A$ and $\alpha_A$ do not depend 
on $\theta$ implies that $q_A^{(1)}(\Delta)=\beta_A^{(1)}(\Delta)=0$\@. 
Therefore, $q_A$ and $\beta_A$ do not depend on $\theta$, but only 
on $\Delta$\@. Notice furthermore
that the factors $p_{ab}$, which determines how strongly the probability of 
the feedback connections is modulated with distance, only enters into the 
expressions for the modulation with $\theta$ of $u_A$, $m_A$, %\alpha_a$,
$q_A$ and $\beta_A$\@. Since these are all 0 when $\zeta_A=0$, in a network 
without functional map the solution is independent of $p_{ab}$\@.

When there is a map, none of these simplifications apply. It is 
however worth noting that in this case, since $m_A^{(0)}$ and $m_A^{(1)}$
both grow proportionally with $m_0$, their ratio in independent of $m_0$\@.
This means that the Circular Variance (\CircVar, see below) of the population
averaged response, which satisfies $\CircVar=1-m_A^{(1)}/m_a^{(0)}$, is 
independent of contrast.



.\\
\newpage
.\\
\newpage

\section{Adding specific connections}
We will now study the effects of introducing a small number $\mathcal{O} (\sqrt{K})$ of PO specific recurrent E-to-E connections. Since the network is operating in the balance regime, this results in a significant $\mathcal{O}(1)$ input to the neurons. 
Let us consider a two population recurrent network of binary neurons where a neuron $i$ in population $A$ has PO $\phi_A^i$. The recurrent network receives external input from FF population which is assumed to consist of neurons with symmetric tuning curves given by, $m^i_0(\theta) = m^{(0)}_0 + \sum_{n=1}^{\infty} m^{(n)}_0 \cos 2n (\theta - \theta_0^i)$, which produces an effective FF input, $u_A^i(\theta) =  J_{A0}  [\sqrt{K} m_0^{(0)} + \sum_{n=1}^{\infty} m^{(n)}_0  z_n^i \cos2 n (\theta - \phi_A^i(n))]$. Making a further assumption that the higher moments are negligible (i.e $m^{(1)}_0 >> m^{(n)}, \forall n > 1$), the effective FF input can be written as $u_A^i(\theta) =  J_{A0}  [\sqrt{K} m_0^{(0)} + m_0^{(1)}z_i \cos2 (\theta - \phi_A^i)]$. Where,  $z_i \sim z e^{-\frac{z^2}{2}}$ and $\phi_A^i \sim \mathcal{U}(0, \pi)$. 
\subsection{Network Architecture} \label{netwArch}
Starting with a random network $\hat{C}$, we will introduce a small number of PO specific connections E-to-E connections. The resulting network $C$ has $\mathcal{O}(\sqrt{K})$ PO specific E-to-E connections out of $K$. For large K ratio of specific to non-specific connections approaches zero as $\frac{1}{\sqrt{K}}$. The network $\hat{C} $ has probability of connections given by, 
\begin{eqnarray}
P(\hat{C}_{EE}^{ij} = 1) =&& \frac{K - 2 \kappa \, \sqrt{K}}{N_E}  \\
P(\hat{C}_{EI}^{ij} = 1) =&& \frac{K}{N_I} \\
P(\hat{C}_{IA}^{ij} = 1) =&& \frac{K}{N_A}; \,\, A \in {E, I} 
\end{eqnarray}
Assuming the wights $J_{AB}^{ij}$ are drawn from some arbitrary distribution $P_J$ with mean $J_{AB} = \left\langle J_{AB}^{ij} \right\rangle_{P_J}$ and finite variance of $J^2_{AB} = \left\langle \left( J_{AB}^{ij} \right)^2 \right\rangle_{P_J}$. \\
Introducing a small number( i.e $ \mathcal{O} \left( \sqrt{K} \right)$) of specific connections of strength in the recurrent E-to-E connections such that, the probability of connections in new matrix $C$ is,
\begin{eqnarray}
P(C_{EE}^{ij} = 1 | \hat{C}_{EE}^{ij} = 0) =&&  \frac{2 \kappa  \sqrt{K}    }{N_E - K - 2 \kappa  \sqrt{K} } \nonumber \\
\times&& \left[ 1 + \cos2(\phi_E^i - \phi_E^j) \right] \,\,\,\, \\
P(C_{EE}^{ij} = 1 | \hat{C}_{EE}^{ij} = 1) =&& 1  \\
\Rightarrow P(C_{EE}^{ij} = 1) = P(C_{EE}^{ij} =&& 1 | \hat{C}_{EE}^{ij} = 0) P(\hat{C}_{EE}^{ij} = 0) \nonumber \\
+ P(C_{EE}^{ij} =&& 1 | \hat{C}_{EE}^{ij} = 1) P(\hat{C}_{EE}^{ij} = 1) \nonumber  \\
= \frac{K}{N_E} \left[1 + \right.&&  \left. \frac{2 \kappa }{\sqrt{K}} \cos 2(\phi_E^i - \phi_E^j) \right]
\end{eqnarray}
Thus, constructing a matrix with this rule results in an Erdos-Renji network with an average in degree of $K$. Furthermore, the E-to-E connections have on average $\kappa \sqrt{K}$ PO specific connections out of the $K$ inbound connections. 
\subsection{Time averaged input statistics}
The time averaged input $u_A^{i} (\theta)$ to neuron $(i, A)$ at stimulus angle $\theta$ is given by, 
\begin{eqnarray}
\label{ueOfTheta}
u_E^{i} (\theta) =&& u^{(0)}_E + \sqrt{\beta_E - \beta_{E0}^{(1)}} x_i +  \sqrt{\beta_{E0}^{(1)}} z_i \cos 2 (\theta - \phi_E^i) \,\,\,\, \,\,\,\,\nonumber \\
+&& \kappa J_{EE} m_E^{(1)} \cos 2 ( \theta  - \phi_E^i) \\
=&& \overline{u}_E(\theta, x_i, \phi_E^i, \kappa) +  \sqrt{\beta_{E0}^{(1)}} z_i \cos 2 (\theta - \phi_E^i) \\
u_I^{i} (\theta) =&& u^{(0)}_I + \sqrt{\beta_I - \beta_{I0}^{(1)}} x_i +  \sqrt{\beta_{I0}^{(1)}} z_i \cos 2 (\theta - \Delta_I^i) \\
=&& \overline{u}_I(\theta, x_i, \phi_I^i) +  \sqrt{\beta_{I0}^{(1)}} z_i \cos 2 (\theta - \phi_I^i)
\end{eqnarray}
Where, $x_i \sim \mathcal{N}(0, 1)$.  As in the random network before, the input consists of un-tuned and tuned components. With specific connections, now there is an additional tuned component of amplitude  $\kappa J_{EE} m_E^{(1)}$. Here, $m_E^{(1)}$ is the first Fourier moment of the mean activity in $\theta$. 
% Where, 
% \begin{eqnarray}
% \alpha_{A} =&& \sum_{B \in \lbrace 0, E, I \rbrace} J_{AB}^2 m_B^{(0)} \\
% \beta_{A}(\Delta) =&& \sum_{B \in \lbrace E, I \rbrace}  \left(J_{AB}\right)^2 q_B^{(0)}(\Delta) \\
% \beta_{A0} (\Delta) =&& \beta_{A0}^{(0)} + \beta_{A0}^{(1)} \nonumber \\
% =&& J^2_{A0} \left[ (m_0^{(0)})^2 + \frac{1}{2}(m_0^{(1)})^2  \cos(4\Delta) \right] \\
% q_A (\theta, \Delta) =&& \int \mathcal{D} x \,  m_A(x, \theta + \Delta)  m_A(x, \theta - \Delta) 
% \end{eqnarray}
Given the time averaged input, after averaging over $z$, the activity of a neuron at angle theta can be written as, 
\begin{eqnarray}
&&m_A^i(\theta) = m_A(x_i, \theta) \nonumber \\
=&& \int_{0}^{\infty} \mathtt{d} z\,  z \mathtt{e}^{-\frac{z^2}{2}} H\left( \frac{-u_A^i(\theta)}{\sqrt{\tilde{\alpha}_A }} \right)  \\
=&& H\left( \frac{-\overline{u}_A }{\sqrt{\tilde{\alpha}_A }}  \right) + \sqrt{\frac{\beta_{A0}^{(1)} }{  \tilde{\alpha}_A + \beta_{A0}^{(1)} \cos^2 2 (\theta - \phi_A^i) }} \nonumber \\ 
\times&&  \exp \left( \frac{\left( \overline{u}_A \right)^2  }{ 2 \tilde{\alpha}_A} \left[ 1 + \frac{\beta_{A0}^{(1)} }{ \tilde{\alpha}_A + \beta_{A0}^{(1)} \cos^2 2 (\theta - \phi_A^i)  } \cos^2 2 (\theta - \phi_A^i) \right] \right) \nonumber \\
\times&& \left[ H \left( - \frac{ \sqrt{\beta_A^{(1)} } \overline{u}_A \cos 2 (\theta - \phi_A^i)} {  \tilde{\alpha}_A + \beta_{A0}^{(1)} \cos^2 2 (\theta - \phi_A^i) } \right) - 1 \right] \cos2(\theta - \phi_A^i) \label{mAiofTheta} 
\end{eqnarray}

\begin{eqnarray}
\tilde{\alpha}_A =&& \alpha_A - \beta_A(\Delta = 0) \\ % - \beta_{A0}^{(0)} - \beta_{A0}^{(1)} \\
\alpha_{A} =&& \sum_{B \in \lbrace 0, E, I \rbrace} J_{AB}^2 m_B^{(0)} \\
\beta_{A}(\Delta) =&&  \beta_{A0} (\Delta) + \sum_{B \in \lbrace E, I \rbrace}  \left(J_{AB}\right)^2 q_B^{(0)}(\Delta) \\
\beta_{A0} (\Delta) =&& \beta_{A0}^{(0)} + \beta_{A0}^{(1)} \nonumber \\
=&& J^2_{A0} \left[ (m_0^{(0)})^2 + \frac{1}{2}(m_0^{(1)})^2  \cos(4\Delta) \right] \\
q_A (\theta, \Delta) =&& \int \mathcal{D} x \,  m_A(x, \theta + \Delta)  m_A(x, \theta - \Delta) 
\end{eqnarray}
$m_A(x, \theta)$ is given by Eq. \ref{mAiofTheta}, which will be used later to generate the tuning curves and for the estimation of OSI distribution of the population.
\subsection{Average population activities}
The population averaged inputs $u_A(\theta) = \left\langle u_A^i(\theta) \right\rangle_{i}$ is given by,\begin{eqnarray}
u_E(\theta) =&& u^{(0)}_E + \frac{1}{\sqrt{2}} J_{E0} m_0^{(1)} \cos 2 \theta \nonumber \\
\qquad&& \qquad+ \kappa J_{EE} m_E^{(1)} \cos 2 \theta \\
=&& u_E^{(0)} + u_E^{(1)} \costh \\
u_I(\theta) =&& u^{(0)}_I + \frac{1}{\sqrt{2}} J_{I0} m_0^{(1)} \cos 2 \theta  \\
=&& u_E^{(0)} + u_I^{(1)} \costh \\
\end{eqnarray}
The mean activity of the system is completely described by the following equations for the first two moments of the mean activities. All the higher moments of the mean activities are zero. 
%\begin{widetext}
%\begin{eqnarray}
%m_E^{(0)} =&&  \int_0^{\pi} \! \frac{\mathtt{d}\theta}{\pi}  H \left( \frac{-u^{(0)}_E - (\frac{1}{\sqrt{2}} J_{E0} m_0^{(1)}  - \kappa  J_{EE} m_E^{(1)} ) \cos 2 \theta}{\sqrt{\alpha_E} } \right)  \label{me0} \\
%m_E^{(1)} =&& \frac{2}{\pi} \int_0^{\pi} \! \mathtt{d} \theta H \left( \frac{-u^{(0)}_E - \frac{1}{\sqrt{2}} J_{E0} m_0^{(1)} \cos 2 \theta - \kappa  J_{EE} m_E^{(1)} \cos 2 \theta}{\sqrt{\alpha_E} } \right) \cos 2 \theta  \label{me1} \\
%m_I^{(0)} =&& \frac{1}{\pi} \int_0^{\pi} \! \mathtt{d} \theta H \left( \frac{-u^{(0)}_I - \frac{1}{\sqrt{2}} J_{I0} m_0^{(1)} \cos 2 \theta}{\sqrt{\alpha_I} } \right)  \label{mi0} \\
%m_I^{(1)} =&& \frac{2}{\pi} \int_0^{\pi} \! \mathtt{d} \theta H \left( \frac{-u^{(0)}_I - \frac{1}{\sqrt{2}} J_{I0} m_0^{(1)} \cos 2 \theta }{\sqrt{\alpha_I} } \right) \cos 2 \theta  \label{mi1}
%\end{eqnarray}
%\end{widetext}
\begin{eqnarray}
m_A^{(0)} =&&  \int_0^{\pi} \! \frac{\mathtt{d}\theta}{\pi}  H \left( \frac{-u_A^{(0)} - u_A^{(1)} \costh}{\sqrt{\alpha_A} } \right)  \label{me0} \\
m_A^{(1)} =&& 2 \int_0^{\pi} \! \frac{\mathtt{d}\theta}{\pi} H \left( \frac{-u_A^{(0)} - u_A^{(1)} \costh}{\sqrt{\alpha_A} } \right) \cos 2 \theta  \label{me1} 
%m_I^{(0)} =&& \frac{1}{\pi} \int_0^{\pi} \! \mathtt{d} \theta H \left( \frac{-u^{(0)}_I - \frac{1}{\sqrt{2}} J_{I0} m_0^{(1)} \cos 2 \theta}{\sqrt{\alpha_I} } \right)  \label{mi0} \\
%m_I^{(1)} =&& \frac{2}{\pi} \int_0^{\pi} \! \mathtt{d} \theta H \left( \frac{-u^{(0)}_I - \frac{1}{\sqrt{2}} J_{I0} m_0^{(1)} \cos 2 \theta }{\sqrt{\alpha_I} } \right) \cos 2 \theta  \label{mi1}
\end{eqnarray}
When, $m_0^{(1)} = 0$, $m_A^{(1)} = 0$ are solutions of this system. In the next section, we will consider the case when $m_0^{(1)} = 0$ and study the system of equations above for finite specific connectivity i.e. $\kappa \ne 0$. 
\subsection{Symmetry breaking: Bump solution} 
\subsubsection{When  $m_0^{(1)} = 0$}
Mean input, \\
\begin{eqnarray}
u_E(\theta) =&& u^{(0)}_E + \kappa J_{EE} m_E^{(1)} \cos 2 \theta \\
u_I(\theta) =&& u^{(0)}_I \label{uITeq} \\
u^{(0)}_A =&& \sqrt{K} \sum_{B \in \lbrace 0, E, I\rbrace} J_{AB} m_B^{(0)}  - T_A\label{uaTeq} 
\end{eqnarray}

Variance of the input,\\
\begin{eqnarray}
\alpha_A =&& \sum_{B \in \lbrace 0, E, I\rbrace} J_{AB}^2 m_B^{(0)} \label{alphaE}
\end{eqnarray}

Evolution of mean rates is given by,\\
\begin{equation}
\tau_A \frac{d}{dt} m_A(\theta, t) = -m_A(\theta, t) + H\left( \frac{-u_A(\theta, t)}{\sqrt{\alpha_A}} \right)
\end{equation}
At the fixed point the mean activities are completely described by, \\
% \begin{eqnarray}
% m_E^{(0)} =&& \frac{1}{\pi} \int_0^\pi d\theta \; H\left( \frac{1 - u_E^{(0)} - u_E^{(1)} \cos(2 \theta)}{\sqrt{\alpha_A}} \right) \label{me0}\\
% m_E^{(1)} =&& \frac{2}{\pi} \int_0^\pi d\theta \; H\left( \frac{1 - u_E^{(0)} - u_E^{(1)} \cos(2 \theta)}{\sqrt{\alpha_A}}  \right) \cos(2 \theta) \label{me1}
% \end{eqnarray}
% \\
\begin{eqnarray}
m_E^{(0)} =&& \frac{1}{\pi} \int_0^\pi d\theta \; H\left( \frac{-u_E(\theta)}{\sqrt{\alpha_A}} \right) \label{me0m10}\\
m_E^{(1)} =&& \frac{2}{\pi} \int_0^\pi d\theta \; H\left( \frac{-u_E(\theta)}{\sqrt{\alpha_A}}  \right) \cos(2 \theta) \label{me1m10} \\
m_I^{(0)} =&&  H\left( \frac{-u^{(0)}_I}{\sqrt{\alpha_I}} \right) \label{mi0m10} \\
m_I^{(1)} =&& 0 
\end{eqnarray}
$m_E^{(0)}$ and $m_I^{(0)}$ are fixed by requiring balance in eq. [\ref{uaTeq}]. The first moment of the mean activity of inhibitory population, $m_I^{(1)}$ is zero because of Eq.[\ref{uITeq}]. Therefore, we can determine $u_I^{(0)}$ from Eq. \ref{mi0m10}. The moments of the mean input to the excitatory population, $u_E^{(0)}$ and $u_E^{(1)}$ can be determined by simultaneously solving Eq. \ref{me0m10} and \ref{me1m10}.
\subsubsection{Behavior of $m_E^{(0)}$ and $m_E^{(1)}$ as $\kappa$ is varied}
In the limit $K \rightarrow \infty$ limit, $m_E^{(0)}$ will remain unchanged as $\kappa$ is varied, which is consistent with the solution $m_E^{(1)} = 0$. As $\kappa$ is increased, at a critical value of $\kappa = \kappa_c$, the system undergoes a super critical pitchfork bifurcation with $m_E^{(1)} = 0$ being the unstable solution and two symmetric solutions, $\pm m_E^{(1)} \neq 0$. The critical value of $\kappa$ for which there is a non zero first moment can be obtained by expanding $H(\cdot)$ at the fixed point(Eq. \ref{me1m10}) for small $m_E^{(1)}$ and integrating,
\begin{eqnarray}
m_E^{(1)} \, F(\kappa) + \left( m_E^{(1)} \right)^3 \,  G(\kappa) + \mathcal{O} \left(\left( m_E^{(1)} \right)^5 \right) = 0
\end{eqnarray}

If, $\exists \kappa_c: F(\kappa_{c}) = 0$, then $\forall \kappa>\kappa_c$, $m_E^{(1)} \neq 0$ solutions are stable if $\sign (G(\kappa_c)) = -1 $ . The expression for $\kappa_c$ is obtained in terms of model parameters and is given by, 
\begin{eqnarray}
%p_c = 3.404
\kappa_{c} &= \frac{ - \sqrt{\alpha_E^{\star (0)}}}{J_{EE} H^{\prime}(h^{\star})}; \,\,\,\, h^{\star} = \frac{- u_E^{\star (0)}}{\sqrt{\alpha_E^{\star (0)}}} 
\end{eqnarray}
For $\kappa > \kappa_c$, the moments of the input components $u_E^{(0)}$ and $u_E^{(1)}$ vary in such a manner as to keep $m_E^{(0)}$ fixed. 
\subsubsection{Stability of the bump solution}
When $K \rightarrow \infty$, 
\begin{eqnarray}
G(\kappa_c)  =&& -\frac{\kappa^3_c J_{EE}^{3}}{\left( \alpha_E^(0) \right)^{\frac{3}{2}}} \left[ \frac{1}{8} H^{\prime \prime \prime} (A_0) - \frac{1}{4} \frac{\left( H^{\prime \prime} (A_0)\right)^2 }{H^{\prime}(A_0)} \right] \\
=&& - \frac{e^{\frac{-A^2_0}{2}}}{\sqrt{2 \pi}} \frac{\kappa^3_c J_{EE}^{3}}{\left( \alpha_E^(0) \right)^{\frac{3}{2}}}  \left[1 + A^2_0 \right] \\
\Rightarrow&& \sign(G(\kappa_c)) = -1
\end{eqnarray}
where, $A_{0} = H^{-1}(m_E^{(0)}))$

\subsection{OSI distribution}
\newcommand{\mZero}{m_E^{(0)}}
\newcommand{\mOne}{m_E^{(1)}}
\newcommand{\zZero}{z^{(0)}}
\newcommand{\zOne}{z^{(1)}}
OSI $(=1 - \CircVar)$ is commonly used to quantify the degree of response selectivity of neurons to external stimuli. The OSI for the $i^{th}$ neuron given its tuning curve $m_A^{i}(\theta)$ is defined as: 
\begin{equation}
\mathtt{OSI}_{i} = \frac{| \zOne_i |}{ \zZero_i}, \,\,\,\, |z| = \sqrt{(\mathrm{Re}(z))^2 + (\mathrm{Im}(z))^2}
% s_{i} = \frac{ \zOne_i }{ \zZero_i}, \,\,\,\, z = \sqrt{(\mathrm{Re}(z))^2 + (\mathrm{Im}(z))^2}
\label{defosi}
\end{equation}

where,\\
\begin{equation}
z_i^{(n)} = \frac{1}{\pi} \int_0^{\pi} \! \mathtt{d} \theta \,  m_A^i(\theta) \, \exp (2 \, n \,  j \, \theta)) \,\,\,\,; \,\,\,\, j = \sqrt{-1}
\end{equation}
Having determined all the order parameters, we can now study the effect of $\kappa$ on the degree of OS of the population. We generate tuning as described in Appendix \ref{appendix1} curves and subsequently the OSI distribution. 

%%% Local Variables:
%%% mode: latex
%%% TeX-master: "main"
%%% End:









%%% Local Variables:
%%% mode: latex
%%% TeX-master: "main"
%%% End:


\section{\label{sec:simulations}Simulations}
In this section, we present the results of numerical simulations of finite networks of binary neurons with specific connectivity as described in section \ref{netwArch}. We have used sequential dynamics with random update. At each time step we randomly pick a neuron $\sigma_A^i$ and update its state to either $0$ or $1$ with probability $P \left(\sigma_A^1(t, \theta) = 1 \,|\, u_A^i(t, \theta) \right) = \Theta(u_A^i - T_A)$. Where total input to the neuron $\sigma_A^i$ at time $t$ is $u_A^i(t, \theta) = \sum_{B \in \lbrace 0, E, I \rbrace} \frac{J_{AB}}{\sqrt{K}} \sum_{j} C_{AB}^{ij} \sigma_B^j(t, \theta)$. And  $T_A$ is the threshold of the population $A$\@. For simplicity, we set all the weights for connected neurons to be equal to $\frac{J_{AB}}{\sqrt{K}}$. $J_{AB}$ is chosen such that it satisfies the balance conditions.
\subsection{Adding $\mathcal{O} (\sqrt{K})$functionally specific connections}
\begin{figure}[b]
\includegraphics[scale=1]{./figs/kappa_vs_mE1_sim.pdf}% Here is how to import EPS art
\caption{\label{fig:simME1vsKappa} Finite size simulations with $m_0^{(0)} = 0.075$. The theory in the limit of infinite network size predicts a bump solution for some $\kappa > \kappa_c$. Finite size networks exhibit qualitatively similar behaviour. When $m_0^{(1)} = 0$, for some $\kappa > \kappa_c$, the system settles to a bump state i.e $m_E^{(1)} > 0$. When $m_0^{(1)} \neq 0$, already for $\kappa = 0$, the system has a stable bump solution. Results from the simulations converge to the finite $K$ solution (dashed line, Appendix \ref{appendix:sim}) as $N$ is increased.}  
\end{figure}


%Introduced $\kappa \sqrt{K}$ new connections that depended on the diferrence in PO's. \\

% %\subsubsection{$\kappa \; \sqrt{K}$, modified model}
% \begin{itemize}
% 	\item Remove $\kappa \sqrt{K}$ connections on average
% 	\item Replace them by new $\kappa \sqrt{K}$ connections that depend on the difference in PO's
% \end{itemize}
% .\\
% \paragraph{Dynamics of the phase}
% \begin{itemize}
%   \item The final position of the bump depends on the initial conditions
%   \item For a finite network, there is a discrete number of fixed points. The drift term approaches zero as $N \rightarrow \infty$
% \end{itemize}

% \paragraph{Virtual rotation} 
% .\\

\subsection{Specific connections with strenghtened connectiones}
% To simulate a more biologically realistic sinario,
% We constructed the E-to-E connectivity matrix  $\kappa \sqrt{K}$ connections on average and replaced them by new $\frac{\kappa}{\omega} \sqrt{K}$ connections of strength $\omega J_{EE}$ that depend on the difference in PO's.

%%% Local Variables:
%%% mode: latex
%%% TeX-master: "main"
%%% End:


\section{Conclusions}
....

\newpage

\appendix
\section*{Appendixes}
%We model a network of layer 2/3 neurons in the primary visual cortex with salt-and-pepper
organization, that receives input from excitatory cells in layer 4. 

We consider two models, a binary network and a network of conductance-based
neurons. The binary network can be studied fully analytically, enabling us to 
understand the mechanism of orientation selectivity in the limit where the 
connectivity is sparse and the average number of inputs, $K$, goes to infinity.
In this model we compare the mechanism for orientation selectivity in the
case where connectivity is solely dependent on distance, with the one where
the connectivity also depends on the difference in preferred orientation of 
the neurons.

We verify that the results derived in the binary network without functional
map also hold in a more realistic conductance-based model. This model has the 
advantage that the relation between the firing and voltage statistics 
can be investigated here, an issue that cannot be addressed in the binary network.
We also investigate in this model the finite sparseness and finite $K$ effects. 
 
In both models the network consists of $N_E$ excitatory and $N_I$ inhibitory 
neurons organized on a ring with  period $\pi$\@. We denote neuron $i$ of 
population $A$ (with $A=E,I$ and $i=1,2,\ldots,N_A$) as neuron $(i,A)$. 
The position on the ring of neuron $(i,A)$ is $\phi_i^A=\pi i/N_A$.

\subsection*{Binary Networks}
\subsubsection*{The model}
In the binary network the neurons are modeled as binary units. They are 
updated sequentially. The state, $\sigma_i^A$, of neuron $(i,A)$ is updated, 
on average, once per time constant $\tau_A$\@. It is set to zero if, at the 
time of the update, its input, $u_i^A$, is below the threshold, $T_A$, 
and set to one otherwise. \\
 \\
{\noindent \bf Architecture:} The input, $u_i^A$, has three 
components: the feedforward input, $u_i^{A0}$, the recurrent excitatory input, 
$u_i^{AE}$, and the recurrent inhibitory feedback, $u_i^{AI}$\@. 

The recurrent excitation and inhibition are given by
\begin{equation}
u_i^{AB}(t)=\frac{J_{AB}}{\sqrt{K}}\sum_{j=1}^{N_B}C_{ij}^{AB}\sigma_j^B(t)
\qquad \qquad (B=E,I),
\end{equation}
where $J_{AB}/\sqrt{K}$, with $J_{AE}>0$ and $J_{AI}<0$, is the contribution 
of an active presynaptic cell to the recurrent input.
The connection matrix $C_{ij}^{AB}$ is randomly chosen: $C_{ij}^{AB}=1$ 
with probability $\frac{K}{N_B}[1+2p\cos2(\phi_i^A-\phi_j^B)]$ and 
$C_{ij}^{AB}=0$ otherwise. The modulation, $p$, $0\leq p \leq 1/2$, determines 
how much the connection probability depends on the distance between neurons 
on the ring. With this connection matrix a neuron receives, on average, input 
from $K$ excitatory and $K$ inhibitory neurons. 

The component, $u_i^{A0}$, represents the input from layer 4 neurons.
We assume that the neuron receives $cK$ feedforward inputs, which have a 
strength $J_{A0}/c\sqrt{K}$\@. We consider two cases: a cortex
with a functional orientation map and a cortex without such a map.

When there is a functional orientation map, the inputs from layer 4 are 
ordered. Cells receive inputs from layer 4 neurons that tend to respond 
maximally at roughly the same stimulus orientation. This results in a total 
feedforward input that has an orientation tuning comparable to that of
individual layer 4 neurons. Furthermore, neighboring layer 2/3 cells receive 
input from layer 4 neurons with similar preferred orientations (POs). 
We model the resulting feedforward input as
\begin{equation}
u_i^{A0}=\sqrt{K}J_{A0}m_0\left[1+\mu\cos 2 (\theta-\phi_i^A)\right],
\end{equation}
where $\theta$ is the stimulus orientation and $m_0$ is the average activity of the
projecting layer 4 neurons, which increases with the contrast. 
The variable $\mu$, $0<\mu<1$, determines how strongly the feedforward input is 
tuned with orientation. 
Note that the external input is maximal for $\theta=\phi_i^A$ and minimal for
$\theta=\phi_i^A+\pi/2$\@.

In the case without a functional orientation map, layer 2/3 neurons receive 
inputs from layer 4 neurons with POs that are randomly 
distributed. As a result, the total feedforward input is almost untuned, 
even if the response of layer 4 neurons is strongly tuned. In fact the tuning 
of the input is reduced by a factor $1/\sqrt{cK}$ compared to tuning of the 
projecting neurons. In this scenario the orientation for which the 
feedforward input is maximal is uncorrelated with the neuron's position. 
We model this by assuming that the external input into the neurons
satisfies
\begin{equation}
u_i^{A0}=\sqrt{K}J_{A0}m_0\left[1+\frac{\xi_A}{\sqrt{K}}
(v_i^A\cos 2\theta+w_i^A\sin 2\theta)\right],
\end{equation} 
Here $v_i^A$ and $w_i^A$ are random variables, independently drawn from a 
Gaussian distribution with mean 0 and variance 1. The average strength of the 
residual tuning in the external input is parametrized by $\xi_A$,  which depends 
on the sharpness of the tuning of the layer 4 neurons.

It is worth noting that the feedforward input 
can alternatively be written as
\begin{equation}
u_i^{A0}=J_{A0}m_0\left[\sqrt{K}+
\xi_A z_i^A\cos 2(\theta-\Delta_i^A)\right]
\end{equation}
where $z_i^A$ is a positive random variable drawn from the distribution 
$p(z)=ze^{-z^2/2}$ and $\Delta_i^A$ is an angle randomly chosen between 0 and 
$\pi$\@. 

\begin{table}
\begin{center}
\begin{tabular}{|l|c|c|c|}
\hline
 & & $A=E$ & $A=I$ \\ 
\hline
general & $T_A$ & 1 & 1 \\
 & $J_{A0}$ & 2 & 1 \\
 & $J_{AE}$ & 1 & 1 \\
 & $J_{AI}$ & 1.5 & 1 \\
 & $p$ & 0.5 & 0.5 \\
\hline
map & $\mu$ & 0.53 & 0.53 \\
\hline
nomap & $\xi_A$ & $\sqrt{15}$ & $\sqrt{15}$ \\
\hline
\end{tabular}
\end{center}
\caption{Parameters for the binary network.}
\label{bin-param:tab}
\end{table}

We study this model in the limit where the connectivity is sparse and $K$ goes 
to infinity, for $m_0=0.025$ (10\% contrast), $m_0=0.05$ (30\% contrast) 
and $m_0=0.075$ (100\% contrast). Unless specified otherwise, the other 
network parameters used are given in Table \ref{bin-param:tab}.

\subsubsection*{Analysis}

The analysis of the networks with and without orientation maps are very similar.
To avoid repeating almost the same arguments twice, it is convenient to 
analyze a network in which the feedforward input is given by 
\begin{equation}
u_i^{A0} = \sqrt{K}J_{A0}m_0\left[1+\mu\cos 2(\theta-\phi_{iA}^0)+
\frac{\xi_A}{\sqrt{K}}z_i^A\cos 2(\theta-\Delta_i^A)\right]
\end{equation}
Results for the network with orientation map are obtained by setting $\xi_A=0$,
while we obtain the results for the network without a map by taking $\mu=0$.

The analysis is similar to that of van Vreeswijk and Sompolinsky 
\cite{vanVreeswijk2005}, to which we refer for further details.\\
 \\
{\noindent \bf Coarse grained analysis:}
Because the strength of the synapses scales as $1/\sqrt{K}$, temporal 
fluctuations in the input do not vanish in the large $K$ limit. 
Furthermore, because  of the randomness of the connectivity and feedforward 
input, the time-average input for neurons of the same population is not the 
same, even for neurons which are very close. 
 
To analyze the network response we calculate the statistics of the different
variables for neurons $(i,A)$ with $\phi_i^A$ between $\phi$ and
$\phi+\delta\phi$ for a sufficiently small $\delta\phi$\@. Averages over this
subpopulation of variables $X_i^A$ are denoted by $\langle X_i^A
\rangle_{\phi,\delta\phi}$\@. For example $\langle u_i^A
\rangle_{\phi,\delta\phi}$ is the average input of neurons $(i,A)$ with 
$\phi<\phi_i^A<\phi+\delta\phi$\@. Because of the rotational and mirror 
symmetry of the system we have that, for a stimulus with orientation 
$\theta$, this average satisfies $\langle u_i^A(t)\rangle_{\phi,\delta\phi}=
u_A(\theta-\phi,t)$, where $u_A$ is symmetric. In fact for all variables $X$ of 
interest the average will satisfy $\langle X_i^A \rangle_{\phi,\delta \phi}=f_A
(\theta-\phi)$ for some symmetric function $f_A$. Thus it is sufficient to 
calculate the averages only for $\phi=0$ at different stimulus angles 
$\theta$\@. For notational convenience we will denote $\langle \cdot 
\rangle_{0,\delta \phi}$ by $\langle \cdot \rangle_{\delta \phi}$\@.

In the steady state, when the activities are constant, we can write the input into
neuron $(i,A)$, for a stimulus with orientation $\theta+\phi_i^A$ as
\begin{equation}
u_i^A(t)=u_{A}(\theta)+\Delta u_i^{A}(\theta)+\delta u_i^{A}(t),
\label{input_terms:eq}
\end{equation}
where $u_{A}$ is the population averaged input and the quenched disorder in $u_i^A$,
$\Delta u_i^{A}$, is the
difference between the averaged input into the neuron and the population 
average, $\Delta u_i^{A}\equiv \langle u_i^A-u_A\rangle_t$, where
$\langle\cdots\rangle_t$ denotes the average over time (the difference
between the time averaged value of $X$ and its population average is 
called the {\it quenched disorder} in $X$\@).
Finally, $\delta u_i^{A}(t)$ represents the temporal fluctuation in the input.

Because $\Delta u_i^A$ and $\delta u_i^{A}(t)$ are composed of many small
contributions, they have Gaussian statistics.
Thus we alternatively can write
\begin{equation}
u_i^A(t)=u_{A}(\theta)+\sqrt{\gamma_A(\theta)}x_i^A(\theta)+
\sqrt{\alpha_A(\theta)-\gamma_A(\theta)}y_i^A(t),
\label{input_terms1:eq}
\end{equation}
where $\gamma_A$ and $\alpha_A-\gamma_A$ are the variance of the quenched disorder
and the temporal fluctuations respectively, and $x_i^A$ and $y_i^A$ are 
Gaussian random variables with mean 0 and variance $\langle [x_i^A]^2\rangle_{\delta \phi}=
\langle [y_i^A(t)]^2\rangle_{\delta \phi}=1$\@. Furthermore, for sufficiently large 
$t-t^\prime$, $y_i^A(t)$ and $y_i^A(t^\prime)$ are uncorrelated,
$\langle y_i^A(t)y_i^A(t^\prime)\rangle_{\delta \phi}\rightarrow 0$, for
$|t-t^\prime|\rightarrow\infty$\@.

In the steady state, the activity of neuron $(i,A)$, $m_i^A=
\langle \sigma_i^A\rangle_t$, is equal to the probability that the
fluctuations bring the input above threshold,
\begin{equation}
m_i^A=\int\!\frac{dy}{\sqrt{2\pi}}\,e^{-y^2/2}
\Theta \left(u_A+\sqrt{\gamma_A}x_i^A+\sqrt{\alpha_A-\gamma_A}y-T_A\right),
\end{equation}
where $\Theta$ is the Heaviside function, $\Theta(x)=1$ for $x>0$ and
$\Theta(x)=0$ for $x<0$\@. This can be written as
\begin{equation}
m_i^A=H\left(\frac{T_A-u_A-\sqrt{\gamma_A}x_i^A}{ \sqrt{\alpha_A-\gamma_A}}
\right).
\label{indiv-rate:eq}
\end{equation}
Here $H$ denotes the cumulative Gaussian distribution,
$H(x)=\frac{1}{\sqrt{2\pi}}\int_x^\infty\!dy\,e^{-y^2/2}$\@.

Because we know the distribution of $x_i^A$, Eqn.~(\ref{indiv-rate:eq}) 
determines, for a stimulus with given contrast and orientation, the distribution 
of the activities for neurons $(i,A)$ with $\phi<\phi_i^A<\phi+\delta\phi$, 
if we know $u_A$, $\alpha_A$ and $\gamma_A$\@. 

However, we also need to take into account that the random variable $x_i^A$
changes when the stimulus parameters are changed. For example, if
$\theta$ is changed to $\theta^\prime$, $x_i^A$ changes to
$x_i^{\prime A}$ and $x_i^{\prime A}$ should be close to
$x_i^A$ if $\theta-\theta^\prime$ is small. Thus the input statistics of population $A$ is fully known if we have a
description of $u_A$ and $\alpha_A$ for any $m_0$ and $\theta$ and if we know the
correlation in the quenched disorder for any pair of stimuli,
$(m_0,\theta)$ and $(m_0^\prime,\theta^\prime)$.  Below we will sketch
the analysis to derive these parameters.

A convenient feature of the binary neuron model is that the average activity,
$m_A\equiv\langle m_i^A\rangle_{\delta\phi}$, given by
\begin{equation}
m_A(\theta)=\int\!\frac{dx}{\sqrt{2\pi}}\,e^{-x^2/2}
H\left(\frac{T_A-u_A(\theta)-\sqrt{\gamma_A(\theta)}x}
{ \sqrt{\alpha_A(\theta)-\gamma_A(\theta)}}\right)
=H\left(\frac{T_A-u_A(\theta)}{\sqrt{\alpha_A(\theta)}}\right),
\label{meanrate:eq}
\end{equation}
is independent of $\gamma_A$\@. Thus if we can express $u_A$ and $\alpha_A$ in 
the mean activities, $m_B$, this together with Eqn.~(\ref{meanrate:eq}) 
determines these quantities.\\
 \\
{\bf Population averaged inputs:}
The term $u_A(\theta)$ in Eqn.~(\ref{input_terms:eq}) is the average 
of the input $u_i^A$, for neurons 
in population $A$ with $0<\phi_i^A<\delta \phi$, when the stimulus orientation is $\theta$.
It is given by 
$u_A(\theta)=\sum_{B=0,E,I}u_{AB}(\theta)$ where
\begin{equation}
u_{A0}(\theta)=\langle u_i^{A0}\rangle_{\delta\phi}= 
\sqrt{K}J_{A0}m_0[1+\mu\cos 2\theta],
\label{uA0:eq}
\end{equation}
while for $A=E,I$
\begin{eqnarray}
u_{AB}(\theta) = \langle u_i^{AB}\rangle_{\delta\phi} & = & \frac{J_{AB}}{\sqrt{K}} \sum_{j=1}^{N_B}
\langle C_{ij}^{AB}\rangle_{\delta\phi}m_B(\theta-\phi_j^B)
\nonumber \\
& = & \sqrt{K}J_{AB}\left[ m_B^{(0)}+p\,m_B^{(1)}\cos 2\theta\right].
\label{uAB:eq}
\end{eqnarray}
Here $m_A^{(k)}$ is the $k$th Fourier moment of $m_A$, $m_A(\theta)=
\sum_{k=0}^\infty m_A^{(k)}\cos 2k\theta$\@.\\
 \\
{\bf Equal time Fluctuations of the Inputs:}
Because the random connection matrices and feedforward inputs 
are independent, the fluctuations in the different components of the 
input are independent so that we can write for $\alpha_A$,
$\alpha_A(\theta)=\sum_{B=0,E,I}\,\alpha_{AB}(\theta)$,
where $\alpha_{AB}(\theta)=
\langle [u_i^{AB}(t)-u_{AB}(\theta)]^2\rangle_{\delta\phi}$\@.

For the feedforward input we have
\begin{equation}  
\alpha_{A0}(\theta) = [J_{A0}\xi_A m_0]^2
\langle[z_i^A\cos 2(\theta-\Delta_i^A)]^2\rangle
= [J_{A0}\xi_A m_0]^2.
\label{alA0:eq}
\end{equation}
Here we have used that $z_i^A$ and $\Delta_i^A$ are independent, 
$\langle [z_i^A]^2\rangle=2$ and $\langle \cos^2 2(\theta-\Delta_i^A)\rangle=1/2$\@.

For the variance of the feedback, $\alpha_{AB}$, with $B=E,I$, we have
(see \cite{vanVreeswijk2005})
\begin{eqnarray} 
\alpha_{AB}(\theta) & = &\frac{J_{AB}^2}{K}\sum_j
\langle\sum_j C^{AB}_{ij}\rangle_{\delta\phi} m_B(\theta-\phi_j^B)
\nonumber \\
 & = & J_{AB}^2[m_B^{(0)}+p\,m_B^{(1)}\cos 2\theta].
\label{alAB:eq}
\end{eqnarray}

{\noindent \bf Balanced solution:}
In the steady state the averaged activities $m_A$ are given by Eqn.~(\ref
{meanrate:eq}), where by combining Eqns.~(\ref{uA0:eq}) and (\ref{uAB:eq}), we can 
write $u_A$ as $u_A(\theta)= u_A^{(0)}+u_A^{(1)}\cos 2\theta$, while with
Eqns.~(\ref{alA0:eq}) and (\ref{alAB:eq}) we can write $\alpha_A$ as 
$\alpha_A(\theta)= \alpha_A^{(0)}+\alpha_A^{(1)}\cos 2\theta$.
The equal time fluctuations $\alpha_A$ are of order 1, while $u_A$ is of order 
$\sqrt{K}$, unless the leading terms in $u_A^{(0)}$ and $u_A^{(1)}$ cancel.
In the large $K$ limit this means that, when this cancellation does not take place,
 $m_A$ goes to either $m_A=0$ or $m_A=1$,
depending on the sign of $u_A$. 
Therefore the only way in which the system can have low, but non-zero, activities is 
if in the leading order of both $u_A^{(0)}$ and $u_A^{(1)}$, the recurrent 
inhibitory input cancels the total excitatory input.

Imposing this requirement for both populations leads to the balanced
solution where, up to a correction term of order $1/\sqrt{K}$, $m_A$ satisfies
$m_A^{(0)}=A_Am_0$, and $m_A^{(1)}=\frac{\mu}{p}A_Am_0$, where
$A_E=\frac{J_{I0}J_{EI}-J_{E0}J_{II}}{J_{EE}J_{II}-J_{EI}J_{IE}}$ and
$A_I=\frac{J_{E0}J_{IE}-J_{I0}J_{EE}}{J_{EE}J_{II}-J_{EI}J_{IE}}$\@.

This determines $\alpha_A$ in the large $K$ limit (Eqn.~(\ref{alAB:eq})). 
But $u_A$ now depends on 
the $1/\sqrt{K}$ corrections of the activity and remains to be determined. 
The average, $u_A^{(0)}$, and modulation, $u_A^{(1)}$, are determined by

\begin{equation}
m_A^{(0)}=\frac{1}{\pi}\int_0^\pi\!d\theta\,
H\left(\frac{T_A-u_A^{(0)}-u_A^{(1)}\cos 2\theta}{\sqrt{\alpha_A(\theta)}}
\right)
\end{equation}
and
\begin{equation}
m_A^{(1)}=\frac{2}{\pi}\int_0^\pi\!d\theta\,
H\left(\frac{T_A-u_A^{(0)}-u_A^{(1)}\cos 2\theta}{\sqrt{\alpha_A(\theta)}}
\right)\cos 2\theta.
\end{equation}

\noindent{\bf Statistics of the quenched disorder:}
The activity $m_i^A(\theta)$ satisfies
\begin{equation}
m_i^A(\theta)=H\left(\frac{T_A-u_A(\theta)-\sqrt{\gamma_A(\theta)}
x_i^A(\theta)}{\sqrt{\alpha_A(\theta)-\gamma_A(\theta)}}\right).
\label{ind-rate:eq}
\end{equation}
Since $x_i^A(\theta)$ is drawn from a Gaussian with mean 0 and variance 1,
this completely determines the distribution of activitie, $\Pr[m_i^A(\theta)]$, 
 if $\gamma_A$ is known. However, to calculate the joint distribution,
$\Pr[m_i^A(\theta_1),m_i^A(\theta_2),\ldots,m_i^A(\theta_n)]$ of the activities
of a neuron, for stimulus angles $\theta_1,\theta_2,\ldots,\theta_n$, we 
need to know the joint statistics of disorder variables, $x_i^A(\theta_1),
x_i^A(\theta_2),\ldots,x_i^A(\theta_n)$\@. Luckily these are Gaussian random
variables, so that their joint statistics are fully determined by the 
cross-correlations, $\langle(x_i^A(\theta_k)x_i^A(\theta_l)\rangle$\@.
We will now determine these correlations.

It is convenient to write the correlations between $x_i^A$ at angle 
$\theta+\Delta$ and $x_i^A$ at angle $\theta-\Delta$ as
$\langle x_i^A(\theta+\Delta)x_i^A(\theta-\Delta)\rangle=
\frac{\beta_A(\theta,\Delta)}
{\sqrt{\gamma_A(\theta+\Delta)\gamma_A(\theta-\Delta)}}$\@. Since
$\langle [x_i^A(\theta)]^2\rangle=1$, we have that 
$\beta_A(\theta,0)=\gamma_A(\theta)$\@.

We introduce a new variable, $q_A$ defined by $q_A(\theta,\Delta)\equiv
\langle m_i^A(\theta+\Delta)m_i^A(\theta-\Delta)\rangle_{\delta\phi}$\@.
This is the joint probability of a neuron being in the active state both
for a stimulus at $\theta+\Delta$ and at $\theta-\Delta$.
It can be calculated using Eqn.~(\ref{ind-rate:eq}), by averaging over 
the correlated Gaussian variables $x_i^A(\theta+\Delta)$ and 
$x_i^A(\theta-\Delta)$\@. After some algebra one obtains
\begin{equation}
q_{A}(\theta,\Delta) = 
\int\!Dx\, H\left(\frac{T_A-u_A^+-\sqrt{\beta_A}x}
{\sqrt{\alpha_A^+-\beta_A}}\right)
H\left(\frac{T_A-u_A^--\sqrt{\beta_A}x}
{\sqrt{\alpha_A^--\beta_A}}\right),
\label{qA:eq}
\end{equation}
where we used the abbreviations, $u_A^\pm=u_{A}(\theta\pm\Delta)$,
$\alpha_A^\pm=\alpha_{A}(\theta\pm\Delta)$ and
$\beta_A=\beta_{A}(\theta,\Delta)$\@.

Following the same logic as for the fluctuations $\alpha_A$, we can write for
the correlations $\beta_{A}$, $\beta_{A}(\theta,\Delta)=
\sum_{B=0,E,I}\beta_{AB}(\theta,\Delta)$, where 
$\beta_{AB}(\theta,\Delta)=
\langle u_i^{AB}(t)u_i^{AB}(t^\prime)\rangle_{\delta\phi}-
\langle u_i^{AB}(t)\rangle_{\delta\phi}
\langle u_i^{AB}(t^\prime)\rangle_{\delta\phi}$ is the 
contribution to the input correlation due to input from population $B$\@.

The contribution to this correlation from the external input is given by
\begin{eqnarray}  
\beta_{A0}(\theta,\Delta) & = &
[J_{A0}\xi_A m_0]^2\langle[z_i^A]^2\cos 2(\theta+\Delta-\Delta_i^A)
\cos 2(\theta-\Delta-\Delta_i^A)\rangle_{\delta\phi} \nonumber \\
 & = &
[J_{A0}\xi_A m_0]^2\langle[z_i^A]^2\rangle_{\delta\phi}
\langle \cos 2(\theta+\Delta-\Delta_i^A)
\cos 2(\theta-\Delta-\Delta_i^A)\rangle_{\delta\phi} \nonumber \\
 & = & [J_{A0}\xi_A m_0]^2 \cos 4\Delta.
\end{eqnarray}

The input correlations due to the two feedback components, $\beta_{AB}$,
with $B=E,I$, depend on $q_{A}$ and is given by
(see \cite{vanVreeswijk2005})
\begin{equation} 
\beta_{AB}(\theta,\Delta)=
J_{AB}^2[q_{B}^{(0)}(\Delta)+pq_{B}^{(1)}(\Delta)\cos 2\theta],
\label{betaA:eq}
\end{equation}
where $q_{B}^{(k)}$ is the $k$th Fourier component in the variable $\theta$ of
$q_{B}$, $q_{B}(\theta,\Delta)=\sum_k q_{B}^{(k)}(\Delta)\cos 2k\theta$\@.

This expresses $\beta_{A}(\theta,\Delta)$ in $q_{A}^{(0)}(\Delta)$ and
$q_{A}^{(1)}(\Delta)$\@. Self-consistent solutions for these are obtained 
by imposing $q_{A}^{(0)}(\Delta)=\frac{1}{\pi}\int_0^\pi\!d\phi\,
q_{A}(\theta,\Delta)$ and $q_{A}^{(1)}(\Delta)=\frac{2}{\pi}
\int_0^\pi\!d\phi\,q_{A}(\theta,\Delta)\cos 2\theta$ where 
$q_{A}(\theta,\Delta)$ is given by Eqn.~(\ref{qA:eq}).

Extending these results to the case where the feedforward activity, $m_0$, is also 
changed is straightforward:
The population averaged input $u_A$ and input variance $\alpha_A$ are now
functions of $m_0$ and $\theta$, and are calculated as before. 
For the correlations we have to consider 2 stimuli specified by variables
($m_0^+$,$\theta+\Delta$) and ($m_0^-$,$\theta-\Delta$) respectively.  
The correlations in the total input are denoted by 
$\beta_A(m_0^+,m_0^-,\theta,\Delta)$, while for the correlations in the 
activity we write $q_A(m_0^+,m_0^-,\theta,\Delta)$\@.
The corelation $\beta_A$ depends on $q_A$ as
$\beta_A(m_0^+,m_0^-,\theta,\Delta)=[J_{0A}\xi_A]^2m_0^+m_0^-\cos 4\Delta+
\sum_{B=E,I}J_{AB}^2[q_B^{(0)}(m_0^+,m_0^-,\Delta)+
pq_B^{(1)}(m_0^+,m_0^-,\Delta)\cos 2\theta]$, where 
$q_A^{(k)}(m_0^+,m_0^-,\Delta)$ is the $k$th Fourier moment in $\theta$
of $q_A(m_0^+,m_0^-,\theta,\Delta)$\@. $q_A$ is still given by
Eqn.~(\ref{qA:eq}), except that now $u_A^\pm=u_A(m_0^\pm,\theta\pm\Delta)$,
$\alpha_A^\pm=\alpha_A(m_0^\pm,\theta\pm\Delta)$ and 
$\beta_A=\beta_A(m_0^+,m_0^-,\theta,\Delta)$\@.
A self-consistency requirement equivalent to that given above for 
$m_0^\pm=m_0$  determines $\beta_A$.\\
 \\
{\noindent \bf Symmetries:}
The connection probabilities are even functions of the difference in 
positions, $\phi_i^A-\phi_j^B$ and the external input is 
symmetric in $\theta-\phi_i^A$\@. This implies that 
$\beta_{A}(m_0,m_0^\prime,\theta,\Delta)=
\beta_{A}(m_0,m_0^\prime,-\theta,-\Delta)$.
As shown above, $\beta_{A}(m_0,m_0^\prime,\theta,\Delta)=
\beta_{A}(m_0,m_0^\prime,-\theta,\Delta)$\@. Together these two symmetries also imply that
$\beta_{A}(m_0,m_0^\prime,\theta,\Delta)=
\beta_{A}(m_0,m_0^\prime,\theta,-\Delta)$\@. Furthermore, under the 
transformation $(\theta,\Delta)\rightarrow (\theta+\pi/2,\Delta-\pi/2)$ the two
input orientations, $\theta_1=\theta+\Delta$ and $\theta_2=\theta-\Delta$, 
transform to $\theta_1 \rightarrow \theta_1$ and 
$\theta_2\rightarrow\theta_2+\pi$\@. With the $\pi$ periodicity of the system 
this implies that $\beta_{A}(m_0,m_0^\prime,\theta,\Delta)=
\beta_{A}(m_0,m_0^\prime,\theta+\pi/2,\Delta-\pi/2)$\@. Finally, if we make the
change, $(m_0,\theta)\rightarrow(m_0^\prime,\theta^\prime)$ and 
$(m_0^\prime,\theta^\prime)\rightarrow(m_0,\theta)$, the correlations are not 
changed either. This implies that 
$\beta_{A}(m_0,m_0^\prime,\theta,\Delta)=\beta_{A}(m_0^\prime,m_0,\theta,-\Delta)$\@.

Taking these symmetries into account we can write 
$\beta_{A}(m_0,m_0^\prime,\theta,\Delta)=\beta_{A}^{(0)}
(m_0,m_0^\prime,\Delta)+$\\$\beta_{A}^{(1)}(m_0,m_0^\prime,\Delta)\cos 2\theta$ as
\begin{equation}
\beta_{A}(m_0,m_0^\prime,\theta,\Delta)=
\sum_{n=0}^\infty \beta_{A}^{(0,2n)}(m_0,m_0^\prime)\cos 4n \Delta
+\beta_{A}^{(1,2n+1)}(m_0,m_0^\prime)\cos 2\theta\cos 2(2n+1) \Delta,
\end{equation}
where $\beta_{A}^{(0,n)}$ and $\beta_{A}^{(1,n)}$ are the $n$th Fourier
components in $\Delta$ of $\beta_{A}^{(0)}$ and $\beta_{A}^{(1)}$
respectively and $\beta^{(n,m)}(m_0,m_0^\prime)=
\beta^{(n,m)}(m_0^\prime,m_0)$\@.

\noindent{\bf The solution in the cases with and without map:}
So far we have considered the solution in the general case. What does this
imply for the network without a map ($\mu=0$) and the case with a map ($\xi_A=0$)?

When there is no map, 
$m_A^{(1)}=u_A^{(1)}=0$ and therefore, from Eqns.~(\ref{alA0:eq}) and 
(\ref{alAB:eq});
$\alpha_A^{(1)}=0$\@. If we now consider Eqns.~(\ref{qA:eq}) and 
(\ref{betaA:eq}) we see that the fact that $u_A$ and $\alpha_A$ do not depend 
on $\theta$ implies that $q_A^{(1)}(\Delta)=\beta_A^{(1)}(\Delta)=0$\@. 
Therefore, $q_A$ and $\beta_A$ do not depend on $\theta$, but only 
on $\Delta$\@. Notice furthermore
that the factor $p$, which determines how strongly the probability of 
connections is modulated with distance, only enters into the expressions for
$u_A^{(1)}$, $m_A^{(1)}$, $q_A^{(1)}$ and $\beta_A^{(1)}$\@. Since these are 
all 0 when $\mu=0$, in a network a without functional map the solution is
independent of $p$\@.

When there is a map, none of these simplifications apply. It is 
however worth noting that in this case, since $m_A^{(0)}$ and $m_A^{(1)}$
both grow proportionally with $m_0$, their ratio in independent of $m_0$\@.
This means that the Circular Variance (\CircVar, see below) of the population
averaged response satisfies $\CircVar=1-\mu/p$,and is independent of contrast.

\subsubsection*{Generating tuning curves}
Setting $\gamma_A(\theta)x_i^A(\theta)$ to $\Delta u_i^A$, we can
write Eqn. (\ref{ind-rate:eq}) as
\begin{equation}
m_i^A(m_0,\theta)=H\left(\frac{T_A-u_A(m_0,\theta)-\Delta u_i^A(m_0,\theta)}
{\sqrt{\alpha_A(m_0,\theta)-\beta_A(m_0,m_0,\theta,0)}}\right),
\label{indi-rate:eq}
\end{equation}
for a feedforward input with mean activity $m_0$ and stimulus angle $\theta$\@.
We have explained how to calculate $u_A$, $\alpha_A$ and $\beta_A$ and that
$\Delta u_i^A(m_0,\theta)$ is a Gaussian random field with
mean 0 and correlations $\langle \Delta u_i^A(m_0,\theta+\Delta)
\Delta u_i^A(m_0^\prime,\theta-\Delta)\rangle=
\beta_{A}(m_0,m_0^\prime,\theta,\Delta)$\@. Thus the statistics of the input, 
and hence of the activity, is fully specified. 

Unfortunately, due to the 
non-linear relation between input and activity, Eqn.~(\ref{indi-rate:eq}),
it is not straightforward to translate this knowledge into meaningful
statements about the properties of the activity of single cells in population 
$A$, such as the distribution of circular variances of the tuning curves at
a given contrast, or how the tuning curves are modified as the contrast is 
changed. 

We use the following approach: we generate the activity of sample neurons 
at different $m_0$ and $\theta$ that are consistent with the calculated 
statistics, and use averaging over these samples to determine the desired 
properties. The method to generate these sample outputs is as follows.

Since $\Delta u_i^{A}$ is $\pi$-periodic in $\theta$, it can be
written as
\begin{equation}
\Delta u_i^{A}(m_0,\theta)=\sum_{n=0}^\infty V^{(n)}(m_0)\cos 2n\theta+
W^{(n)}(m_0)\sin 2n\theta,
\end{equation}
where $V^{(n)}$ and $W^{(n)}$ are Gaussian random variables with a mean of 0,
whose correlations should be chosen such that 
$\langle \Delta u_i^{A}(m_0,\theta+\Delta)
\Delta u_i^{A}(m_0^\prime,\theta-\Delta)\rangle=
\beta_A(m_0,m_0^\prime,\theta,\Delta)$\@. 

After some straightforward, but tedious algebra one finds that
the correlations have to satisfy 
$\langle V^{(n)}(m_0)W^{(m)}(m_0^\prime)\rangle=0$ and
$\langle V^{(n)}(m_0)V^{(m)}(m_0^\prime)\rangle=
 \langle W^{(n)}(m_0)W^{(m)}(m_0^\prime)\rangle=$\\$
\frac{1}{2}(1+\delta_{n,m})\beta_{A}^{(|n-m|,n+m)}(m_0,m_0^\prime)$\@.
Here $\delta_{n,m}$ is the Kronecker delta, $\delta_{n,m}=1$ for $n=m$ and
$\delta_{n,m}=0$ for $n\neq m$\@.

In principle one would need infinitely many random variables $V^{(n)}$ and 
$W^{(n)}$, but in practice one can get a very good approximation by using a 
rather small number, since $\langle [V^{(n)}(m_0)]^2\rangle=
\langle [W^{(n)}(m_0)]^2\rangle=\beta_{A}^{(0,2n)}(m_0,m_0)$, which rapidly 
decreases as $n$ is increased. Thus the 
amplitude of the higher frequency components in the quenched disorder is 
increasingly small. As a result, setting terms with $n$ larger than some cut 
off $n_0$ has no noticeable effect on the output statistics.
For the parameters we use in this paper we can take $n_0$ as low as $n_0=5$\@.

To generate tuning curves for a neuron of population $A$, for 
$k_0$ contrasts, corresponding to input activities $m_0=m_{0k}$, for 
$k=1,2,\ldots,k_0$ we determine $u_A^{(n)}(m_{0k})$ and 
$\alpha_{A}^{(n)}(m_{0k})$ for each $k$ and $n=0,1$ and calculate
$\beta_A^{(0,2n)}(m_{0k},m_{0l})$ and 
$\beta_A^{(1,2n+1)}(m_{0k},m_{0l})$ for $k,l \in \{1,2,\ldots,k_0\}$
and $n\in \{0.1.\ldots,n_0\}$\@. Next we construct a $N\times N$ correlation 
matrix, $C$, where $N=k_0(n_0+1)$, in which the elements, $C_{i,j}$ satisfy
$C_{k_0n+k,k_0m+l}=\frac{1}{2}(1+\delta_{n,m})\beta_A^{(|n-m|.n+m)}
(m_{0k},m_{0l})$ if $|n-m|\leq 1$ and $C_{k_0n+k,k_0m+l}=0$ otherwise. 
We use Cholesky decomposition \cite{Horn1985} to find the lower triangular matrix 
$L$ which satisfies $LL^{\sf T}=C$\@. We construct two $N$ dimensional vectors, 
$\vec{x}$ and $\vec{y}$ whose elements are independently drawn from a Gaussian 
distribution with mean 0 and variance 1.
From these we calculate the vectors $\vec{v}=L\vec{x}$ and $\vec{w}=L\vec{y}$\@.
Using this procedure we have that, on average $\langle\vec{v}\vec{v}^{\sf T}
\rangle=\langle\vec{w}\vec{w}^{\sf T}\rangle=C$ and $\langle\vec{v}
\vec{w}^{\sf T}\rangle=0$\@.
Therefore, if we set $V_k^{(n)}=v_{k_0n+k}$ and $W_k^{(n)}=w_{k_0n+k}$, the 
time averaged quenched disorder, $\Delta u_i^A(m_0,\theta)$, will have the desired 
statistics. 

Using Eqn.~(\ref{indi-rate:eq}) we can now calculate the neuronal output for
different angles $\theta$ and different input levels $m_0$, to generate the
orientation tuning curves for a sample neuron at different contrasts.
More samples can be produced by applying this algorithms to many sets
of vectors, $\vec{x}$ and $\vec{y}$, drawn independently.
 
\section{\label{appendix2} Model with specific connections}
We follow similar steps as in Appendix \ref{appendix1} to compute the order parameters of the network with specific connections. 
\subsection{Mean input}
The population averaged mean input is defined as, 
\begin{eqnarray}
u_A(\theta) :=&& \lim_{K \rightarrow \infty} \lim_{N_0 \rightarrow \infty} \PopAvgSqr{\PopAvg{u_A^i(\theta, t)}{t}}{\delta \theta}  \\
=&& u_A^{(0)} + u_A^{(1)} \cos(2 \theta)
\end{eqnarray}
The recurrent components of the mean input is given by, 
\begin{eqnarray}
u_{A, rec}^{(n)} = \sqrt{K} \; (1 &&+ \Theta(n-1)) \sum_{B \in \lbrace E, I  \rbrace} J_{AB} \nonumber \\
\times&&  \int_0^{\pi} \! \frac{d\theta^{\prime}}{\pi} \left[ 1 +  \frac{2\kappa_{AB}}{\sqrt{K}} \; \cos(2 \theta^{\prime}) \right] \nonumber \\
\qquad \qquad \qquad \qquad && \times  m_B(\theta - \theta^{\prime}) \cos( 2 n \theta) 
\end{eqnarray} \\
Using $m_B(\theta) = \sum_{n = 0}^{\infty} m_B^{(n)} \cos 2 n \theta$ , $\kappa_{EE} = \kappa$ and, $\kappa_{IE} = \kappa_{EI} = \kappa_{II} = 0$, we get
\begin{eqnarray}
u_A^{(0)} =&& \sqrt{K} \left[J_{A0} m_0^{(0)} + \sum_B J_{AB} m_B^{(0)} \right] \label{ue0equation} \\
u_E^{(1)} =&& \frac{1}{\sqrt{2}} J_{E0} m_0^{(1)} \cos 2 \theta + \kappa J_{EE} m_E^{(1)} \cos 2 \theta \label{ue1} \\
u_I^{(1)} =&& \frac{1}{\sqrt{2}} J_{E0} m_0^{(1)} \cos 2 \theta  \label{ui1}
\end{eqnarray}

\subsection{Variance of the input}
The population averaged variance of the total input is given by, 
\begin{eqnarray}
\alpha_A :=&& \lim_{K \rightarrow \infty} \lim_{N \rightarrow \infty}
\PopAvgDPhi{\PopAvg{(u_A^i(\theta, t))^2}{t}} \nonumber \\ 
-&& \PopAvgDPhi{\PopAvg{(u_A^i(\theta, t))}{t}}^2  \\
=&& J_{A0}^2 m_0^{(0)} + \sum_{B \in \lbrace  E, I \rbrace } J_{AB}^2 \nonumber \\
\times&&\int_0^{\pi} \! \frac{d\theta^{\prime}}{\pi} \left[ 1 +  \frac{2\kappa_{AB}}{\sqrt{K}} \; cos(2 \theta^{\prime}) \right] m_B(\theta - \theta^{\prime}) \\
=&& J_{A0}^2 m_0^{(0)} + \sum_{B \in \lbrace  E, I \rbrace } J_{AB}^2 m_B^{(0)} \label{alphaEapdx}
\end{eqnarray}

\subsection{Quenched disorder}
The population averaged variance of the time averaged inputs is given by, 
\begin{eqnarray}
\beta_A(\theta,&& \Delta)  :=  \nonumber \\
&&\lim_{K \rightarrow \infty} \lim_{N_0 \rightarrow \infty} \left\lbrace \left[ \PopAvg{u^i_A(\phi + \Delta, t) - \PopAvgDPhi{u^i_A(\phi + \Delta, t)}}{t} \right. \right. \nonumber \\
\qquad&& \left. \left. \PopAvg{u^i_A(\phi - \Delta, t)- \PopAvgDPhi{u^i_A(\phi - \Delta, t)}}{t} \right]_{\delta \phi} \right\rbrace \\ 
=&& J^2_{A0} \left[ (m_0^{(0)})^2 + \frac{1}{2}(m_0^{(1)})^2  \cos(4\Delta) \right] + \sum_{B \in \lbrace  E, I \rbrace } J_{AB}^2 \nonumber \\
\times&& \int_0^{\pi} \! \frac{d\theta^{\prime}}{\pi} \left[ 1 +  \frac{2\kappa_{AB}}{\sqrt{K}} \; cos(2 \theta^{\prime}) \right] q_B(\theta, \Delta) \\
=&& J^2_{A0} \left[ (m_0^{(0)})^2 + \frac{1}{2}(m_0^{(1)})^2  \cos(4\Delta) \right] \nonumber \\
+&& \sum_{B \in \lbrace E, I \rbrace}  \left(J_{AB}\right)^2 q_B^{(0)}(\Delta)\\
=&&  \beta_{A0} (\Delta) + \sum_{B \in \lbrace E, I \rbrace}  \left(J_{AB}\right)^2 q_B^{(0)}(\Delta)
\end{eqnarray}
.
\subsection{Computing $\kappa_{critical}$}
In the large $K$ limit, $m_E^{(0)}$ will remain unchanged as $\kappa$ is varied.
\begin{eqnarray}
H(z) &:= \frac{1}{2}  erfc\left(\frac{z}{\sqrt{2}} \right) 
\end{eqnarray}
Expanding $H(\cdot)$ for small $m_E^{(1)}$ in Eq. \ref{me1m10}, %where, \\ $x = -\Hx$ and $b = 
\begin{widetext}
\begin{eqnarray}
\label{mE1Int}
m_E^{(1)} =&& \frac{2}{\pi} \int_0^\pi d\theta \cos(2\theta) \left\lbrace H \Hb - C_0 m_E^{(1)} \cos(2\theta)  +  C_1 \left( m_E^{(1)} \cos(2\theta) \right)^2  - C_2  \left( m_E^{(1)} \cos(2\theta) \right)^3 \right. \nonumber \\
&&\qquad \qquad \qquad \qquad\qquad\qquad\qquad\quad+ \left.  \mathcal{O} \left( (m_E^{(1)})^4 \right) \right\rbrace \\
C_0 =&& H^{\prime}\Hb  \left( \Hx \right) =  -\CZero  \left( \Hx \right) \\ 
C_1 =&&  H^{\prime \prime}\Hb  \left( \Hx \right)^2 = \COne \left( \Hx \right)^2 \\ 
C_2 =&& H^{\prime \prime \prime}\Hb  \left( \Hx \right)^3 = -\CTwo \left( \Hx \right)^3
\end{eqnarray}
\end{widetext} 
Integrating Eq.[\ref{mE1Int}], \\
\begin{eqnarray}
m_E^{(1)} =&& -\frac{\kappa J_{EE}} {\sqrt{\alpha_E}} m_E^{(1)} H^{\prime} \Hb \nonumber \\
+&& \mathcal{O} \left( (m_E^{(1)})^3 \right) \label{me1Sol} \\
m_E^{(1)} &&\left[1 + \frac{\kappa J_{EE}} {\sqrt{\alpha_E}} H^{\prime} \Hb \right] = 0 
\end{eqnarray}
\begin{eqnarray}
\kappa_c =&& \frac{ - \sqrt{\alpha_E}}{J_{EE} H^{ \prime } \Hb}
\end{eqnarray}
\subsection{Stability of the bump solution}
\newcommand{\avar}{\left( \frac{\kappa J_{EE}}{\sqrt{\alpha_E}} \right)}
\begin{widetext}
\begin{eqnarray}
\label{stabme0}
\frac{1}{\sqrt{\alpha_E}} \left(1 - \uEZero - \uOne \costh \right)  =&& A_0 + \mEOne (A_1 - \avar \costh) + \left( \mEOne \right)^2 A_2  \nonumber \\
\qquad\qquad\qquad\qquad&& + \left( \mEOne \right)^3 A_3 + \mathcal{O} \left( (\mEOne)^4 \right) \\
\mEZero = H(A_0) + \mEOne \Hp A_1 + \left( \mEOne \right)^2 && \left[A_2  \Hp + \Hpp \left(\frac{A^2_1}{6} + \frac{1}{4} \avar^2 \right) \right] \nonumber \\
+  \left( \mEOne \right)^3 \left[A_3 \Hp + A_2 A_1 \Hpp \right. +&& \left. \Hppp \left( \frac{A^3_1}{6} + \frac{A_1}{4}  \right) \avar^3 \right] + \mathcal{O} \left( (\mEOne)^4 \right)\\
\mEOne = - \mEOne \Hp \avar -  \left( \mEOne \right)^2&& \left[A_1 \avar  \Hpp \right] \nonumber \\
\qquad -  \left( \mEOne \right)^3 \left[A_2 \avar \Hpp \right. +&& \left. \Hppp \left( \frac{A^2_1}{2} + \frac{1}{8} \avar^3  \right) \avar^3 \right] + \mathcal{O} \left( \mEOne \right)^4 
\label{stabme1}
\end{eqnarray}
\end{widetext}
When $K \rightarrow \infty$, balance requires that  $\mEZero$ and $\mIZero$ must remain fixed even when $mEOne \neq 0$. This is true only when all the coefficients of $\mEOne$ in Eq. \ref{stabme0} are zero, therefore, \\

\begin{eqnarray}
A_0 =&& H^{-1} \left( \mEZero \right) \\
A_1 =&& 0 \\
A_2 =&& \frac{-1}{4} \avar^2 \frac{\Hpp}{\Hp}\\
A_3 =&& 0
\end{eqnarray}
 Eq. \ref{stabme1} reduces to, 
\begin{eqnarray}
\mEOne =&& - \mEOne \Hp \avar  \\
+&&  \left( \mEOne \right)^3 \left[- A_2 \avar \Hp \right. \nonumber \\
-&& \left.  \avar^3  \frac{\Hppp}{8}  \right] \\ 
= \mEOne F(\kappa, \mEZero) +&& \left( \mEOne \right)^3 G(\kappa, \mEZero) \\
G(\kappa_c)  = -\frac{\kappa^3_c J_{EE}^{3}}{\left( \alpha_E^(0) \right)^{\frac{3}{2}}}&& \left[ \frac{1}{8} H^{\prime \prime \prime} (A_0) - \frac{1}{4} \frac{\left( H^{\prime \prime} (A_0)\right)^2 }{H^{\prime}(A_0)} \right] \\
=&& - \frac{e^{\frac{-A^2_0}{2}}}{\sqrt{2 \pi}} \frac{\kappa^3_c J_{EE}^{3}}{\left( \alpha_E^(0) \right)^{\frac{3}{2}}}  \left[1 + A^2_0 \right] \\
\Rightarrow&& \sign(G(\kappa_c)) = -1
\end{eqnarray} 

\subsection{\label{appendix:sim}Simulations}
The numeric curve is obtained by solving the following equations simultaneously, without taking the limit $K \rightarrow \infty$ 
\begin{widetext}
\begin{eqnarray}
\label{me0Sol}
m_E^{(0)}(\kappa) =&& \frac{1}{\pi} \int_0^\pi d\theta \;  H\left( \frac{1 - \sqrt{K} \sum_B J_{AB} m_B^{(0)} - (2 ^ {-\frac{1}{2}} J_{E0} m_0^{(1)} + \kappa J_{EE} \mEOne(\kappa) )\cos(2 \theta) }{\sqrt{\sum_B J_{EB}^2 m_B^{(0)} + \frac{\kappa}{\sqrt{K}} J^2_{EE} \mEOne(\kappa) \cos2\theta} } \right)\\
\label{me1Sol}
\mEOne(\kappa) =&& \frac{2}{\pi} \int_0^\pi d\theta \; \cos 2\theta H\left( \frac{1 - \sqrt{K} \sum_B J_{AB} m_B^{(0)} - (2 ^ {-\frac{1}{2}} J_{E0} m_0^{(1)} + \kappa J_{EE} \mEOne(\kappa)) \cos(2 \theta)}{\sqrt{\sum_B J_{EB}^2 m_B^{(0)} + \frac{\kappa}{\sqrt{K}} J^2_{EE} \mEOne(\kappa) \cos2\theta} } \right)  \\
\label{mi0Sol}
m_I^{(0)} =&& \frac{1}{\pi} \int_0^\pi d\theta \;  H\left( \frac{1 - \sqrt{K} \sum_B J_{AB} m_B^{(0)} - 2 ^ {-\frac{1}{2}} J_{I0} m_0^{(1)} \cos 2\theta }{\sqrt{\sum_B J_{IB}^2 m_B^{(0)}}} \right) \\
m_I^{(1)} =&& \frac{2}{\pi} \int_0^\pi d\theta \; \cos 2 \theta  H\left( \frac{1 - \sqrt{K} \sum_B J_{AB} m_B^{(0)} - 2 ^ {-\frac{1}{2}} J_{I0} m_0^{(1)} \cos 2\theta }{\sqrt{\sum_B J_{IB}^2 m_B^{(0)}}} \right) \\
\end{eqnarray}
\end{widetext}

%%% Local Variables:
%%% mode: latex
%%% TeX-master: "main"
%%% End:




% The \nocite command causes all entries in a bibliography to be printed out
% whether or not they are actually referenced in the text. This is appropriate
% for the sample file to show the different styles of references, but authors
% most likely will not want to use it.
% \nocite{*}
\bibliography{apssamp}% Produces the bibliography via BibTeX.

\end{document}
%
% ****** End of file apssamp.tex ******

%%% Local Variables:
%%% mode: latex
%%% TeX-master: t
%%% End:
